\chapter{Permutations and Combinations}
\begin{definition}{}{}
  The terms \(n\) \emph{pick} \(r\) and \(n\) \emph{choose} \(r\) respectively denote 
  \[{^n}P_r \coloneq  \frac{n!}{(n-r)!} \qquad\text{and}\qquad \binom{n}{r}\coloneq{^n}C_r\coloneq \frac{n!}{(n-r)!r!}.\]
\end{definition}
\begin{stbox}{General Information}
  \begin{itemize}
    \item Addition and multiplication principles
    \item Know how to `bundle' objects together so as to calculate the total no. of permutations.
    \item There are 
    \[\frac{n!}{n_1!n_2!\cdots n_r!}\]
    number of ways to arrange \(n\) objects, of which \(n_i\) are similar, for each \(i\).
  \end{itemize}
\end{stbox}
    \begin{fact}
      Intuition: If there are \(n_1\) objects are non-distinct out of \(n\) objects, then there are \(n_1!\) ways to arrange these objects that results in `the same' permutation.
    \end{fact}
  \begin{stbox}{}
  \begin{itemize}
    \item Case-wise considerations/calculations (then summing together the total number of permutations)
    \item Unordered circular permutations:\\
    There are \(n!/n=(n-1)!\) number of ways of arranging \(n\) distinct objects in a circle.
  \end{itemize}
\end{stbox}
    \begin{fact}
      For unordered circular permutations, we do not care if you rotate the seating arrangement, as long as neighbours are preserved for each object. i.e. \((A,B,C,D) \sim (B,C,D,A)\). As a result, each such collection of \(n\) permutations reduces down to one. Thus, explaining the division by \(n\).
    \end{fact}
\begin{stbox}{}
\begin{itemize}
    \item Complementary Method, i.e. taking number of arrangements without restriction - number of arrangements with the opposite of that restriction.
\end{itemize}
\end{stbox}
    \begin{example}{}{}
      Number of ways two girls \emph{cannot} sit next to each other = number of arrangements \emph{without restriction} \(-\) number of arrangements with girls sitting \emph{together}.
    \end{example}
\begin{stbox}{}
\begin{itemize}
    \item Insertion Method, place down some of your objects and then insert the rest in the gaps.
  \end{itemize}
\end{stbox}
    \begin{example}{}{}
      \begin{enumerate}[label={}]
        \item Boys sit at table first: \(2!\) ways.
        \item \vspace{-1mm} From the 3 gaps, choose 2 for the 2 girls to sit at: 3 ways.
        \item \vspace{-1mm} The girls can arrange themselves in \(2!\) ways.
        \item \vspace{-1mm} So, total no. of ways is \(2! \cdot 3 \cdot 2!=12\).
      \end{enumerate}
    \end{example}
\begin{stbox}{}
\begin{itemize}
    \item Ordered circular permutations: First calculate the number of unordered permutations, then add the ordering at the end.
  \end{itemize}
\end{stbox}
    \begin{note}
      Circular arrangements are not the same as row arrangements.
      
      We know that \(A\) and \(B\) are not considered to be seating together in the row arrangement of \((A,C,D,E,B)\). But, they are seating together in a corresponding row arrangement. The number of row arrangements can be less than, equal to, or more than the number of circular arrangements.
    \end{note}