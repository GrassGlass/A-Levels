\chapter{Inequalities and Equations}
\section{Solving Inequalities}
\begin{stbox}{General Methods}
  \begin{enumerate}
    \item Quadratic formula for factorisation / finding roots (of polynomial).
    \item Completing the square.
    \item Discriminant/Completing the square to eliminate factors which are \emph{always} positive or negative (e.g. removing \(x^2-3x+4\)). \emph{Note to include coefficient of \(x^2\) in the argument.}
    \item GC (include sketch).
    \item \emph{Rational Functions}: Move everything to one side by adding or subtracting, then use a number line.
\end{enumerate}
\end{stbox}
\section{Modulus Inequalities}
\begin{fact}
    Given \(x \in \mathbb{R}\), we have that 
    \begin{itemize}
        \item \(\lvert x \rvert \geq 0\),
        \item \(\lvert x^2 \rvert=\lvert x \rvert^2=x^2\),
        \item \(\sqrt{x^2}=\lvert x \rvert\).
    \end{itemize}
    And as long as \(x \in \mathbb{R}^+\),
    \begin{itemize}
        \item \(\sqrt{x}^2=\lvert x \rvert\).
    \end{itemize}
\end{fact}
\begin{stbox}{Useful Properties}
  For every \(x,k \in \mathbb{R}\):
  \begin{enumerate}[label=(\alph*)]
      \item \(\lvert x \rvert < k\) iff \(-k<x<k\).
      \item \(\lvert x \rvert > k\) iff \(x<-k\) or \(x>k\).
  \end{enumerate}
\end{stbox}
\section{Summary}

\begin{GCSkills}{}
  \begin{enumerate}
    \item Plotting curves \(y=f(x)\) in G.C.
    \item How to use simultaneous equation solver.
  \end{enumerate}
\end{GCSkills}
\begin{IN}
  \begin{itemize}
    \item Eliminating Factors --- \emph{only} works for \(c=0\) in \(f(x) \geq c\) or \(f(x) \leq c\).

    Counterexample: It is false that \(P(x)=x(3x^2-9x+10) \leq 2\) iff \(x \leq 2\). Notice that \(P(1.8)=6.336 \not\leq 2\). 
    \item Discriminant --- include coefficient of \(x^2\) in argument.
    \item When using factor elimination to remove some \(f(x)\), we only need to say that ``\(f(x)\) is negative''.
    \item Rational functions --- exclude the values that causes division by zero to occur.
    \item With inequalities, be really careful about multiplication! If \(x>y\) \emph{and} \hly{\(z>0\)}, then \(xz>yz\). 
    \item Cross multiplication preserves order for \(\frac{x}{y}<\frac{x'}{y'}\) iff \(y\) and \(y'\) are \emph{both} positive or negative.
    
    Note the counterexample \(\frac{1}{2}<\frac{1}{-3}\).
    \item Squaring preserves/reverses order for \(x<y\) iff \(x\) and \(y\) are \emph{both} positive or negative.
    \item Can't necessarily use differentiation to solve if qns asks for algebraic method.
    \item Safer to graph out the two functions separately!
    \item Be careful about whether to include equality! Don't forget to account for it!
    \item  Note that when solving for \(\lvert x \rvert=y\), \(\lvert x \rvert < y\), etc, \(y\) must be greater than or equal to 0. In other words, there may be solutions we will need to reject.
    \item Carelessness: Look at the question carefully! If they ask for a \emph{set} of values, then rmb to give it as a \emph{set}!
    \item Exponentiation and Logarithms: Simply use \(\ln\) and avoid \(\log_c\) for \(c<1\).
    
    Order is \emph{Preserved} under exponentiation/logarithms if the base is \emph{larger than} one. Otherwise, when it is \emph{less than} one, the order is \emph{reversed}. \url{https://www.desmos.com/calculator/gd8z5fa0bg}
    \item  For more complicated real-world-context qns, try playing around with the values (e.g. use simultaneous equations) first. It may work out nicer than expected.
\end{itemize}
\end{IN}