\chapter{Graphing Techniques}
\section{Graphing `Familiar' Functions and Asymptotic bois}

\begin{definition*}{}{}
  \begin{enumerate}
    \item \textbf{Lines of Symmetry}: A \emph{line of symmetry} of a function is a line, such that the function is a reflection of itself about that line.
    \item \textbf{Horizontal Asymptotes}: A (horizontal) line \(g(x)=c\) is the \emph{horizontal asymptote} of the curve \(f(x)\) iff \(\lim_{x \to \infty}{f(x)}=c\) (or with \(-\infty\) instead of \(\infty\)).\footnote{Otherwise notated by \(f(x) \to c\) as \(x \to \infty\).}
    \item \textbf{Vertical Asymptotes}: A (vertical) line \(x=c\) is a \emph{vertical asymptote} of the curve \(f(x)\) iff \(\lim_{x \to c}{f(x)}=\operatorname{\infty} \text{ or } -\infty\).
    \item \textbf{Oblique Asymptotes}: A line \(g(x)=mx+c\) --- where \(m \neq 0\) --- is an \emph{oblique asymptote} of the curve \(f(x)\) iff \(\lim_{x \to \infty}[f(x)-g(x)]=0\) (or with \(-\infty\) instead of \(\infty\)).
\end{enumerate}
\end{definition*}
\begin{stbox}{Curve Sketching of Rational Functions}
  \begin{enumerate}
    \item[\textbf{S}] Stationary points
    \item[\textbf{I}] Intersection with axes
    \item[\textbf{A}] Asymptotes   
  \end{enumerate}
  \begin{enumerate}[label=\roman*]
    \item Know how to sketch the graphs of \(y=\dfrac{ax+b}{cx+d}\) and \(y=\dfrac{ax^2+bx+c}{dx+e}\).
    \item Rectangular Hyperbolas \(\left(\text{of the form \(y=\dfrac{ax+b}{cx+d}\)}\right)\):
    \begin{itemize}
      \item \emph{Two} asymptotes, namely \(x=-\dfrac{d}{c}\) and \(y=\dfrac{a}{c}\).
      \item \emph{Two} lines of symmetry with gradients \(\pm 1\) \emph{and} pass through the intersection point of the aforementioned two asymptotes.
    \end{itemize}
    \item \emph{If} \(n=\operatorname{deg}P=\operatorname{deg}Q\), then
    \begin{itemize}
      \item \(y=R(x)\) is the \emph{horizontal} asymptote of \(\dfrac{P(x)}{Q(x)}=R(x)+\dfrac{S(x)}{Q(x)}\).
      \item Equivalently, \(y=\dfrac{\operatorname{coeff}_P(x^n)}{\operatorname{coeff}_Q(x^n)}\) is a \emph{horizontal} asymptote.\footnote{E.g.: \(y=\dfrac{1}{15}\) is a horizontal asymptote of \(y=\dfrac{\text{\hly{\(1\)}}x^2+2x-3}{(\text{\hly{\(5\)}}x+1)(\text{\hly{\(3\)}}x+2)}\).}
    \end{itemize}
    \item If \(\operatorname{deg}P=\operatorname{deg}Q+1\), then \(R(x)\) is an \emph{oblique} asymptote of \(\dfrac{P(x)}{Q(x)}=R(x)+\dfrac{S(x)}{Q(x)}\).
    \item Write down asymptotes and lines of symmetry.\footnote{E.g.: \begin{itemize}
      \item[] Asymptotes: \(x=4\), \(y=20\).
      \item[] Lines of Symmetry: \(y=x+16\), \(y=-x+24\).
    \end{itemize} } If none are present indicate with ``No lines of symmetry.''
  \end{enumerate}
\end{stbox}
\begin{IN}
  \begin{itemize}
    \item The discriminant can be very useful.
    \item Be aware of how to use the G.C. Transfrm app.
    It allows you to vary the value of some parameter \(A\) for a function \(f(Ax)\). Use this to graphically find the values of \(k\) that satisfy some condition(s).
  \end{itemize}
\end{IN}

\section{Conics}
``Tikz is pain, PGFPlots is suffering'' --- Wise Man.
\begin{center}
  \small
  \begin{tabular}{|c|c|c|}
    \hline
    & Ellipses & Hyperbolas\\
    \hline
    Standard Forms & \(\dfrac{(x-h)^2}{a^2}+\dfrac{(y-k)^2}{b^2}=1\) & 
    \begin{tabular}{@{}c@{}} 
     \\
    \(\dfrac{(x-h)^2}{a^2}-\dfrac{(y-k)^2}{b^2}=1\)\\
    \(\dfrac{(y-k)^2}{b^2}-\dfrac{(x-h)^2}{a^2}=1\)\\
    \\
    \end{tabular}\\
    \hline
    General Equation & 
    \begin{tabular}{@{}c@{}} 
      \(ax^2+by^2+cx^2+dx+e=0\),\\
      \footnotesize where \(\operatorname{sgn}(a)=\operatorname{sgn}b\). \normalsize
     \end{tabular}
     &
     \begin{tabular}{@{}c@{}} 
      \(ax^2+by^2+cx^2+dex+e=0\),\\
      \footnotesize where \(\operatorname{sgn}(a) \neq \operatorname{sgn}b\). \normalsize
     \end{tabular}\\
     \hline
     Center & \multicolumn{2}{c|}{\((h,k)\)}\\
    \hline
    \begin{tabular}{@{}c@{}} 
      Vertical `Radius'\\
      \footnotesize (variables here from \emph{standard form}!) \normalsize
    \end{tabular}
      & \multicolumn{2}{c|}{\(b\)}\\
      \hline
    \begin{tabular}{@{}c@{}} 
        Horizontal `Radius'\\
        \footnotesize (variables here from \emph{standard form}!) \normalsize
    \end{tabular}
    & \multicolumn{2}{c|}{\(a\)}\\
    \hline
    \begin{tabular}{@{}c@{}} 
      Vertical  Vertices\\
      \footnotesize (variables here from \emph{standard form}!) \normalsize
    \end{tabular}
    & \multicolumn{2}{c|}{\((h, k \pm b)\)}\\
    \hline
    \begin{tabular}{@{}c@{}} 
      Horizontal Vertices\\
      \footnotesize (variables here from \emph{standard form}!) \normalsize
    \end{tabular}
    & \multicolumn{2}{c|}{\((h \pm a,k)\)}\\
    \hline
    Shape & 
    \begin{tikzpicture}[scale=0.5]
      
      \begin{axis}[axis lines=middle,axis line style =-{Classical TikZ Rightarrow[length=5pt 3 0]},every axis x label/.style = {%
        at = {(xticklabel cs:1.05)},
        anchor = north},
      every axis y label/.style = {%
        at = {(yticklabel cs:1.05)},
        anchor=east},
        xtick=\empty, ytick=\empty,clip=false,xmin=0,xmax=11,xlabel=\Large\(x\),ylabel=\Large\(y\),ymin=0,ymax=6
        ]
        
      \draw[blue] (5,3) ellipse (5 and 2);
    
      % label center
      \node at (5,3) [below right] {\Large \((h,k)\)};

      \node at (5,3) {\Large \color{red}{\(\times\)}};
    
      % draw h-line
      \draw[->,arrows = -{Classical TikZ Rightarrow[length=5pt 3 0]}] (5,3) -- (0,3) node[midway, above] {\Large \(a\)};
    
      % draw another h-line
      \draw[->,arrows = -{Classical TikZ Rightarrow[length=5pt 3 0]}] (5,3) -- (10,3);
    
      % draw k-line
      \draw[->,arrows = -{Classical TikZ Rightarrow[length=5pt 3 0]}] (5,3) -- (5,5) node[midway, right] {\Large \(b\)};
    
      % draw another k-line
      \draw[->,arrows = -{Classical TikZ Rightarrow[length=5pt 3 0]}] (5,3) -- (5,1);

      \addplot+[
  mark=x,
  only marks,
  mark size=6pt,
  mark options={line width=1.5pt,red}
] 
  coordinates
  {(5,3)};
  
      \end{axis}
      % draw ellipse
      \addvmargin{3mm}
    \end{tikzpicture} & 
    \begin{tabular}{@{}c@{}} 
      \(\operatorname{coeff}(x^2)<0\)\\
      \begin{tikzpicture}[scale=0.5]
      
        \begin{axis}[axis lines=middle,axis line style =-{Classical TikZ Rightarrow[length=5pt 3 0]},every axis x label/.style = {%
          at = {(xticklabel cs:1.05)},
          anchor = north},
        every axis y label/.style = {%
          at = {(yticklabel cs:1.05)},
          anchor=east},
          xtick=\empty, ytick=\empty,clip=false,xmin=-1,xmax=11,xlabel=\Large\(x\),ylabel=\Large\(y\),ymin=-1,ymax=7
          ]
          
      \addplot [red,thick,domain=-2:2] ({sinh(x)+5}, {cosh(x)+3});
      \addplot [red,thick,domain=-2:2] ({-sinh(x)+5}, {-cosh(x)+3});
      \addplot[red,dashed,domain=1:9] {x-2};
      \addplot[red,dashed,domain=1:9] {-x+8};

        \node at (5,3) [below] {\Large \((h,k)\)};
  
        \node at (5,3) {\LARGE  \color{blue}{\(\times\)}};
      
        % draw h-line
        \draw[->,arrows = -{Classical TikZ Rightarrow[length=3pt 3 0]}] (5,4) -- (4,4) node[midway, above] {\Large \(a\)};
      
        % draw k-line
        \draw[->,arrows = -{Classical TikZ Rightarrow[length=3pt 3 0]}] (5,3) -- (5,4) node[midway, right] {\Large \(b\)};
        
        \addplot+[
  mark=x,
  only marks,
  mark size=6pt,
  mark options={line width=1.5pt}
] 
  coordinates
  {(5,3)};

        \end{axis}
        % draw ellipse
      \end{tikzpicture}\\
      \(\operatorname{coeff}(y^2)<0\)\\
      \begin{tikzpicture}[scale=0.5]
      
        \begin{axis}[axis lines=middle,axis line style =-{Classical TikZ Rightarrow[length=5pt 3 0]},every axis x label/.style = {%
          at = {(xticklabel cs:1.05)},
          anchor = north},
        every axis y label/.style = {%
          at = {(yticklabel cs:1.05)},
          anchor=east},
          xtick=\empty, ytick=\empty,clip=false,xmin=-1,xmax=11,xlabel=\Large\(x\),ylabel=\Large\(y\),ymin=-1,ymax=7
          ]
          
        \addplot [red,thick,domain=-2:2] ({cosh(x)+5}, {sinh(x)+3});
      \addplot [red,thick,domain=-2:2] ({-cosh(x)+5}, {sinh(x)+3});
      \addplot[red,dashed,domain=1:9] {x-2};
      \addplot[red,dashed,domain=1:9] {-x+8};

        \node at (5,3) [below=0.25cm] {\Large \((h,k)\)};
  
        \node at (5,3) {\LARGE  \color{blue}{\(\times\)}};
      
        % draw h-line
        \draw[->,arrows = -{Classical TikZ Rightarrow[length=3pt 3 0]}] (5,3) -- (4,3) node[midway, above] {\Large \(a\)};
      
        % draw k-line
        \draw[->,arrows = -{Classical TikZ Rightarrow[length=3pt 3 0]}] (4,3) -- (4,4) node[midway, left] {\Large \(b\)};

        \addplot+[
  mark=x,
  only marks,
  mark size=6pt,
  mark options={line width=1.5pt}
] 
  coordinates
  {(5,3)};

        \end{axis}
        % draw ellipse
      \end{tikzpicture}
    \end{tabular}\\
    \hline
    \begin{tabular}{@{}c@{}} 
      Asymptotes\\
      (No need to rmb!)
    \end{tabular}
    & - & \(y=k \pm \dfrac{b(x-h)}{a}\)\\
    \hline
    Lines of Symmetry & \multicolumn{2}{c|}{\(x=h\), \(y=k\)}\\
    \hline
  \end{tabular}
  \normalsize
\end{center}
\begin{stbox}{General Information}
  \begin{itemize}
    \item To find asymptote of hyperbolas, solve  
    \[\frac{(x-h)^2}{a^2}=\frac{(y-k)^2}{b^2}.\]
    \item When sketching any conic, label its vertices or radii, together with its center and asymptotes.
  \end{itemize}
\end{stbox}
\section{Parametric Equations}
\begin{IN}
  \begin{itemize}[label=\(\star\)]
    \item Check the qns for any \emph{restrictions} on the parameter! And modify that of the G.C.'s accordingly (Tmin \& Tmax).
    \item Vary the \(t\)-step or resolution (when using cartesian coordinates) when the graph is oddly jagged.
  \end{itemize}
\end{IN}
\section{Scaling, Translations, and Reflections}

\begin{center}
  \begin{tabular}{|Sc|Sc|Sc|}
    \hline
    \multicolumn{3}{|Sc|}{\large \color{blue}{Playing With \(x\)}}\\
    \hline
    Function & \(x\) is replaced with & (Horizontal) Transformation\\
    \hline 
    \(f(x+a)\) & \(x+a\) & 
    \begin{tabular}{@{}c@{}} 
      Translate \(a\) units in the positive (\(a \leq 0\))\\ 
      O/R negative \(x\)-direction (\(a \geq 0\)).
    \end{tabular}\\
    \hline 
    \(f(-x)\) & \(-x\) & Reflect about the \(y\)-axis\\
    \hline
    \(f(ax)\) & \(ax\) & Scale parallel to the \(x\)-axis by a scale factor of \(\dfrac{1}{a}\) if \(a \geq 0\).\\
    \hline
    \multicolumn{3}{|Sc|}{\large \color{red}{Playing With \(f(x)\)}}\\
    \hline
    \multicolumn{2}{|c|}{Function / Change to \(f(x)\)} & (Vertical) Transformation\\
    \hline
    \multicolumn{2}{|c|}{\(f(x)+a\)} & \begin{tabular}{@{}c@{}} 
      Translate \(a\) units in the positive (\(a \geq 0\))\\ 
      O/R negative \(y\)-direction (\(a \leq 0\)).
    \end{tabular}\\
    \hline
    \multicolumn{2}{|c|}{\(-f(x)\)} & Reflect about the \(x\)-axis.\\
    \hline
    \multicolumn{2}{|c|}{\(af(x)\)} & Scale parallel to the \(y\)-axis by scale factor \(a\).\\
    \hline
  \end{tabular}
\end{center}
\begin{IN}
\begin{center}
  \begin{tikzpicture}
    \node[ellipse, draw=blue, thick, minimum size=10mm,align=center] (x) at (1,0) {Transform \(x\)\\[2mm]
    Translation \ding{239} Scaling / Reflection};
    \node[ellipse, draw=red, thick, minimum size=10mm,align=center] (y) at (1,-3) {Transform \(y\)\\[2mm]
    Scaling / Reflection \ding{239} Translation};
    \draw [black, line width=0.75pt, arrows = {-Latex[open,scale=2]}] (x) to (y);
  \end{tikzpicture}
\end{center}
\end{IN}
\newpage
\section{\(\lvert f(x) \rvert\) and \(f( \lvert x \rvert)\)}
\begin{stbox}{General Information}
  \begin{itemize}
    \item For \(\lvert f(x) \rvert\), simply flip the part of the graph of \(f(x)\) that is below the \(x\)-axis, to above the \(x\)-axis.
    \item For \(f(\lvert x \rvert)\), its graph is symmetric about the \(x\)-axis
  \end{itemize}
\end{stbox}
\section{\(y=\frac{1}{f(x)}\)}
\begin{center}
  \begin{tabular}{|Sc|Sc|}
    \hline
    Behavior of \(f(x)\) & Behavior of \(1/f(x)\)\\
    \hline
    \(f(x)>0\) & \(\dfrac{1}{f(x)}>0\)\\
    \hline 
    \(f(x)<0\) & \(\dfrac{1}{f(x)}<0\)\\
    \hline
    Vertical Asymptote at \(x=c\) & \begin{tabular}{@{}c@{}} 
      \(\dfrac{1}{f(x)}\) \emph{tends} to 0\\
      \scriptsize \(^{*}\dfrac{1}{f(x)}\) is undefined at \(x=c\) \normalsize\\
    \end{tabular}\\
    \hline
    \multicolumn{2}{|Sc|}{\begin{tabular}{@{}c@{}} 
      \(\dfrac{df}{dx}=-\dfrac{d}{dx}\mathopen{}\left(\dfrac{1}{f(x)}\right)\)\\
      \scriptsize i.e. when \(f(x)\) increases, \(\dfrac{1}{f(x)}\) decreases. \normalsize\\
    \end{tabular}}\\
    \hline 
    \((a,b)\) is a \emph{minimum} pt & \(\left(a,\dfrac{1}{b}\right)\) is a \emph{maximum} pt\\
    \hline
    \((a,b)\) is a \emph{maximum} pt & \(\left(a,\dfrac{1}{b}\right)\) is a \emph{minimum} pt\\
    \hline
  \end{tabular}
\end{center}