\chapter{Sampling and Estimation}
\begin{definition}{}{}
  A sample is a finite subset of the population.
\end{definition}
\begin{definition}{}{}
  A random sample is a sample selected such that each member of the population has an equal probability of being selected into the sample.
\end{definition}
\begin{note}
  State, in context, what it means for the sample to be random.
  \begin{center}
    \parbox{0.9\textwidth}{
      It means that \hly{every [a member of the population]} has \hly{an equal probability} of being \hly{selected into the sample}. 
    }
  \end{center}
\end{note}
\begin{note}
  Explain why the sample would actually not be random.
  \begin{center}
    \parbox{0.9\textwidth}{
      [Contextual reason], so \hly{not all} the [members of the population] have an \hly{equal probability of being selected into the sample}. 
    }
  \end{center}
\end{note}
\begin{definition}{}{}
  Any statistic \(T\) derived from a random sample and used to estimated an unknown population parameter \(\theta\) is known as an \emph{estimator}. It is an \emph{unbiased} estimator iff \(\E(T)=\theta\). If \(T\) is unbiased we commonly write \(\hat{\theta}\) for \(T\).
\end{definition}
\begin{stbox}{General Information}
  \begin{itemize}
    \item Either write \(\hat{\mu}\highlight[yellow]{=\bar{x}}=\dots\) or write out ``Unbiased estimate of the population mean \(\mu\), \(\widebar{x}=\dots\)'' Same holds for other population parameters \(\theta\).
    \item Estimators you should know:
    \item 
    % *A table where Parameter is under another column
    % \begin{center}
    %   \begin{tabular}{|Sc|Sc|Sc|Sc|Sc|}
    %     \hline
    %     \multicolumn{2}{|Sc|}{Parameter} & Estimator & Unbiased? & Formula\\
    %     \hline
    %     Population Mean & \(\mu\) & \(\widebar{X}\) & \checkmark & \(\dfrac{X_1+X_2+\dots+X_n}{n}\)\\
    %     \hline
    %     \multirow{2}{*}[-1.2cm]{Population Variance} & \multirow{2}{*}[-1.2cm]{\(\sigma^2\)} & \(\sigma_n^2\) & \(\times\) & 
    %     \begin{minipage}{3cm}
    %       \begin{center}
    %         \(\dfrac{\sum{(X_i-\widebar{X})^2}}{n}\)\\[1mm]
    %         \(\dfrac{\sum{X_i^2}}{n}-\widebar{X}^2\)
    %       \end{center}
    %     \end{minipage}\\
    %     \cline{3-4}
    %     & & \(S^2\) & \checkmark & 
    %     \begin{minipage}{5cm}
    %       \begin{center}
    %       \(\dfrac{n}{n-1}\sigma_n^2\)\\[1mm]
    %       \(\dfrac{\sum(X_i-\widebar{X})^2}{n-1}\)\\[1mm]
    %       \(\dfrac{1}{n-1}\left[ \sum X_i^2-\dfrac{(\sum X_i)^2}{n} \right]\)
    %       \end{center}
    %     \end{minipage}\\
    %     \hline
    %     Population Proportion & \(p\) & \(P_s\) & \checkmark & \(\dfrac{X}{n}\)\\
    %     \hline
    %   \end{tabular}
    % \end{center}
    \begin{center}
      \resizebox{0.94\textwidth}{!}{\begin{tabular}{|Sc|Sc|Sc|Sc|}
        \hline
        Parameter & Estimator & Unbiased? & Formula(s)\\
        \hline
        Population Mean \(\mu\) & Sample Mean \(\widebar{X}\) & \checkmark & \(\dfrac{X_1+X_2+\dots+X_n}{n}\)\\
        \hline
        \multirow{2}{*}[-1.2cm]{Population Variance \(\sigma^2\)} & Sample Variance \(\sigma_n^2\) & \(\times\) & 
        \begin{minipage}{3cm}
          \begin{center}
            \(\dfrac{\sum{(X_i-\widebar{X})^2}}{n}\)\\[1mm]
            \(\dfrac{\sum{X_i^2}}{n}-\widebar{X}^2\)
          \end{center}
        \end{minipage}\\
        \cline{2-4}
        & \(S^2\) & \checkmark & 
        \begin{minipage}{5cm}
          \begin{center}
          \(\dfrac{n}{n-1}\sigma_n^2\)\\[1mm]
          \(\dfrac{\sum(X_i-\widebar{X})^2}{n-1}\)\\[1mm]
          \(\dfrac{1}{n-1}\left[ \sum X_i^2-\dfrac{(\sum X_i)^2}{n} \right]\)
          \end{center}
        \end{minipage}\\
        \hline
        Population Proportion \(p\) & Sample Proportion \(P_s\) & \checkmark & \(\dfrac{X}{n}\)\\
        \hline
      \end{tabular}}
    \end{center}
    \item Let \(X\) be a random variable following \emph{any distribution}, and suppose we have a random sample \(X_1,X_2,\dots,X_n\) of size \(n\geq 50\). Then by CLT (Central Limit Theorem), since \(n\geq 50\) is large, 
    \[\widebar{X}\sim \Normal\left(\mu,\frac{\sigma^2}{n}\right) \qquad\text{and}\qquad X_1+X_2+\dots+X_n\sim \Normal(n\mu,n\sigma^2)\]
    \emph{approximately}.
    \item Assumptions when using CLT:
    \begin{itemize}
      \item The sample is random.
      \item Each \(X_i\) is independent and identically distributed.
    \end{itemize}
    \item Suppose \(X\sim \Normal(\mu,\sigma^2)\) is known and we pick a \emph{particular} sample. Then,
    \begin{center}
      \begin{tabular}{|Sc|Sc|Sc|}
        \hline
        Distribution & Is An Approximation?\\
        \hline
        \(\widebar{X} \sim \Normal(\mu,\sigma^2)\) & No\\
        \hline
        \(\widebar{X} \sim \Normal(\widebar{x},\sigma^2)\) & Yes\\
        \hline
        \(\widebar{X} \sim \Normal(\mu,s^2)\) & Yes\\
        \hline
        \(\widebar{X} \sim \Normal(\widebar{x},s^2)\) & Yes\\
        \hline
      \end{tabular}
    \end{center}
    % ADD NAMES of the parameters and estimators to previous table
    So, if we obtain any of the latter three in solving a question, we must write ``\(X\sim \Normal(\rule{3mm}{0.1mm},\rule{3mm}{0.1mm})\) approximately'' (even though we knew \(X\) \emph{exactly} follows a normal distribution!)
    \item Pooled estimators. First assume we have two populations, from which we select a random sample of size \(n_1\) and \(n_2\). We let \(\widebar{X}_1\) and \(S_1^2\) denote the sample mean and unbiased estimator for variance, respectively, for the first sample. Similarly define \(\widebar{X}_2\) and \(S_2^2\), for the second sample.
    \begin{center}
      \begin{tabular}{|Sc|Sc|}
        \hline
        Parameter & Unbiased Pooled Estimator\\
        \hline
         Mean  & \(\hat{\mu}=\dfrac{n_1\widebar{X}_1+n_2\widebar{X}_2}{n_1+n_2}\)\\
         \hline
         Variance & \(S_p^2=\dfrac{(n_1-1)S_1^2+(n_2-1)S_2^2}{n_1+n_2-2}\)\\
         \hline
      \end{tabular}
    \end{center}
  \end{itemize}
\end{stbox}
The following definition is found in 
\href{https://www.amazon.com/Introduction-Mathematical-Statistics-8th-Whats-dp-0134686993/dp/0134686993/ref=dp_ob_title_bk}{Hogg-McKean-Craig}. Similar definitions are also found in 
\href{https://www.amazon.sg/Mathematical-Statistics-Applications-William-Mendenhall/dp/0495110817#customerReviews}{Wackerly-Mendenhall-Schaefer}
and \href{https://www.amazon.com/Probability-Statistical-Inference-Statistics-Monographs/dp/0824703790}{Nitis Mukhopadhyay}.
\begin{definition}{}{}
  Let \(X_1,X_2,\dots,X_n\) be a sample on a random variable \(X\), where \(X\) has pdf \(f(x;\theta)\), \(\theta \in \Omega\). Let \(0<\alpha<1\) be specified. Let \(L=L(X_1,X_2,\dots,X_n)\) and \(U=U(X_1,X_2,\dots,X_n)\) be two statistics. We say that the interval \((L,U)\) is a \((1-\alpha)100\%\) \emph{confidence interval} for \(\theta\) iff 
  \[1-\alpha=P_\theta[\theta \in (L,U)].\]
  That is, the probability that the interval contains \(\theta\) is \(1-\alpha\), which is called the \emph{confidence coefficient} or \emph{confidence level} of the interval.
\end{definition}
\begin{stbox}{}
  \begin{itemize}
    \item We cannot write ``a \(1-\alpha\) (e.g. 0.95) confidence interval''. The \(1-\alpha\) must always be expressed as a \emph{percentage}.
    % \item Let \(0<\alpha<1\) and \(X_1,X_2,\dots,X_n\) be a sample on a random variable \(X\). Suppose \(n\) is large (\(n\geq 50\)). Then, CLT allows us to obtain the approximation
    % \[\frac{\widebar{X}-\mu}{\frac{S}{\sqrt{n}}}=Z\sim \Normal(0,1).\]
    % Rewriting \(\Prob(-z_{1-\alpha/2}<Z<z_{1-\alpha/2})=1-\alpha\) gives
    % \[\Prob\left( \widebar{x}-z_{1-\alpha/2}\frac{s}{\sqrt{n}}<\mu<\widebar{x}+z_{1-\alpha/2}\frac{s}{\sqrt{n}}\right)=1-\alpha.\] 
    % As such, a \((1-\alpha)100\%\) confidence interval is
    % \[\left( \widebar{x}-z_{1-\alpha/2}\frac{s}{\sqrt{n}}\,,\ \widebar{x}+z_{1-\alpha/2}\frac{s}{\sqrt{n}} \right).\]
    % When the variance \(\sigma^2\) is known, we can replace \(s\) with \(\sigma\).
    \item Let \(\hat{\theta}\) be a statistic that is normally distributed with mean \(\theta\) and standard error \(\sigma_{\hat{\theta}}\). We see that 
    \[\frac{\hat{\theta}-\theta}{\sigma_{\hat{\theta}}}=Z \sim \Normal(0,1).\]
    Rewriting \(\Prob(-z_{1-\alpha/2}<Z<z_{1-\alpha/2})=1-\alpha\) gives
    \[\Prob(\hat{\theta}-z_{1-\alpha/2}\sigma_{\hat{\theta}}<\theta<\hat{\theta}+z_{1-\alpha/2}\sigma_{\hat{\theta}})=1-\alpha.\]
    Hence, a \((1-\alpha)100\%\) confidence interval for \(\theta\) is
    \[(\hat{\theta}-z_{1-\alpha/2}\sigma_{\hat{\theta}}\,,\ \hat{\theta}+z_{1-\alpha/2}\sigma_{\hat{\theta}}).\]
    \href{https://www.amazon.sg/Mathematical-Statistics-Applications-William-Mendenhall/dp/0495110817#customerReviews}{(Wackerly-Mendenhall-Schaefer)}
    \item Let \(0<\alpha<1\) and \(X_1,X_2,\dots,X_n\) be a sample on a random variable \(X\) with population mean \(\mu\), where \(n\) is large. Then, an approximate \((1-\alpha)100\%\) confidence interval for \(\mu\) is
    \[\left( \widebar{x}-z_{1-\alpha/2}\frac{s}{\sqrt{n}}\,,\ \widebar{x}+z_{1-\alpha/2}\frac{s}{\sqrt{n}} \right).\]
    When the population variance \(\sigma^2\) is known, we can replace \(s\) with \(\sigma\). If the distribution of \(X\) is known to be normal, in addition to \(\sigma^2\) being known exactly, then the confidence interval is exact; it is not just an approximation. 

    \href{https://www.amazon.com/Introduction-Mathematical-Statistics-8th-Whats-dp-0134686993/dp/0134686993/ref=dp_ob_title_bk}{(Hogg-McKean-Craig)}
    \item Let \(X\) be a Bernoulli random variable with probability of success \(p\), where \(X\) is 1 or 0 if the outcome is success or failure, respectively. Suppose \(X_1,X_2,\dots,X_n\) is a random sample from the distribution of \(X\), where \(n\) is large. Let \(\hat{p}=\widebar{X}\) be the sample proportion of successes. Then, an approximate \((1-\alpha)100\%\) confidence interval for \(p\) is given by 
    \[\left( \hat{p}-z_{1-\alpha/2}\sqrt{\frac{\hat{p}(1-\hat{p})}{n}}\,,\ \hat{p}+z_{1-\alpha/2}\sqrt{\frac{\hat{p}(1-\hat{p})}{n}} \right).\]
    (Letting \(Y=X_1+X_2+\dots+X_n\sim \operatorname{B}(n,p)\) gives \(\hat{p}=Y/n\), which is the presentation used in the school's notes.) 

    \href{https://www.amazon.com/Introduction-Mathematical-Statistics-8th-Whats-dp-0134686993/dp/0134686993/ref=dp_ob_title_bk}{(Hogg-McKean-Craig)}
  \end{itemize}
\end{stbox}
\begin{note}
  Standard phrasing for the interpretation of a \((1-\alpha)100\%\) confidence interval \((a,b)\). 
  \begin{center}
    \parbox{0.9\textwidth}{
      The probability that the interval \((a,b)\) contains the true value of the [population mean/proportion in context] is \(1-\alpha\).
    }
  \end{center}
\end{note}
\begin{note}
  Standard phrasing for what is a \((1-\alpha)100\%\) confidence interval for \(\theta\)?
  \begin{center}
    It is an interval which has probability \(1-\alpha\) of containing the true value of \(\theta\).  
  \end{center}
\end{note}
\begin{note}
  Standard phrasing for whether a population parameter \(\theta\) has likely increased/decreased, when given suitable confidence intervals.
  \begin{enumerate}
    \item There is no conclusive result.

    As the old \((1-\alpha)100\%\) confidence interval has a lower/higher central value of \(\widehat{\theta}_1<\widehat{\theta}_2\) than the new \((1-\alpha)100\%\) confidence interval, there is some evidence to suggest that [\(\theta\) in context] has increased/decreased. However, this evidence is weak: Since the old and new \((1-\alpha)100\%\) confidence intervals overlap, we are unable to conclude whether the [\(\theta\) in context] has decreased or not. Hence, it is inconclusive from these figures as to whether the [context (e.g. an awareness campaign)] has been effective.
    \item It has likely increased/decreased.
    
    The old \((1-\alpha)100\%\) confidence interval is to the left/right of the new \((1-\alpha)100\%\) confidence interval, such that they do not overlap. So, can conclude that the [\(\theta\) in context] likely increased/decreased. Hence, these figures suggests that the [context (e.g. an awareness campaign)] has been effective.
  \end{enumerate}
\end{note}
\begin{note}
  Advantage and disadvantage of a \((1-\beta)100\%\) confidence interval compared to a \((1-\alpha)100\%\) confidence interval, where \(\beta<\alpha\).
  \begin{center}
    \begin{tabular}{m{0.15\textwidth}m{0.8\textwidth}}
      \toprule
      Advantage:& A \((1-\beta)100\%\) CI is more likely to contain the true mean. So, any result made on the assumption that the true mean is contained in the \((1-\beta)100\%\) CI is more likely to hold true, than results made assuming the true mean is contained in the \((1-\alpha)100\%\) CI.\\
      \midrule
      Disadvantage:& A \((1-\beta)100\%\) CI is wider. Thus, using the \((1-\beta)100\%\) CI gives us less accuracy about the location of the true value of the mean. Accordingly, stronger results can be concluded by assuming that the true mean is contained in the \((1-\alpha)100\%\) CI, than can be concluded by assuming the true mean is contained in the \((1-\beta)100\%\) CI.\\
      \bottomrule
    \end{tabular}
  \end{center}
  \emph{Note.} Clearly state which is the advantage and disadvantage, as illustrated above.
\end{note}
\begin{GCSkills}{}
  Calculating statistics (i.e. \(\widebar{x}\), \(s\), etc) by G.C. given data for a sample.
  \begin{enumerate}
    \item Keying in the data: \texttt{stat} \(\Longrightarrow\) \texttt{1:Edit} \(\Longrightarrow\) Key in the data into one of the lists \(\texttt{L}_i\). 
    \item Calculating the statistic: \texttt{stat} \(\Longrightarrow\) \texttt{CALC} \(\Longrightarrow\) \texttt{1-Var Stats} (\texttt{List:}\(\texttt{L}_i\)) \(\Longrightarrow\) \texttt{Calculate}.
    \item Getting the statistic for further calculations: \texttt{vars} \(\Longrightarrow\) \texttt{5:Statistics} \(\Longrightarrow\) Select the desired statistic.
  \end{enumerate}
\end{GCSkills}
\begin{GCSkills}{}
  Calculating the symmetric confidence interval for a normally distributed random variable.
  \begin{center}
  \begin{tabular}{ll}
    Mean: & \texttt{stat} \(\Longrightarrow\) \texttt{TESTS} \(\Longrightarrow\) \texttt{7:ZInterval\dots}\\
    Proportion: & \texttt{stat} \(\Longrightarrow\)\texttt{TESTS} \(\Longrightarrow\) \texttt{A:1-PropZInt\dots}\\ 
  \end{tabular}
  \end{center}
\end{GCSkills}
\begin{titlednote}{combining two samples into one.}
  Let \(X\) be a random variable with population mean \(\mu\). We are given two independent random samples \(A\) and \(B\), of sizes \(m\) and \(n\), respectively. Find a \(100(1-\alpha)\%\) confidence interval for \(\mu\), based on the combined sample \(C\). 
  \begin{align*}
    \widebar{x}_C&=\frac{\sum{x_A}+\sum{x_B}}{m+n}=\frac{m\widebar{x}_A+n\widebar{x}_B}{m+n}\\
    s_C^2&=\highlight[green!50]{\frac{1}{m+n-1}\left( \sum{x_A^2}+\sum{x_B^2}-\frac{(\sum{x_A}+\sum{x_B})^2}{m+n} \right)}\neq\highlight[red!30]{\frac{(m-1)s_A^2+(n-1)s_B^2}{m+n-2}}.
  \end{align*}
  Proceed with the usual calculations to find the \(100(1-\alpha)\%\) confidence interval based on \(C\).
\end{titlednote}