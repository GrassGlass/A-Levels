\chapter{Series and Sequences}
\section{Binomial Theorem and Series}
\begin{theorem}{The Binomial Theorem}{}
  \[(a+b)^n=\sum_{r=0}^{n}{\binom{n}{r}a^{n-r}b^r},\]
  where \(n \in \mathbb{Z}^{+}\).
\end{theorem}
\begin{theorem}{The Binomial Series}{}
  \[(1+x)^{p}=\sum_{r=0}^{\infty}{\binom{p}{r}x^r},\]
  where \(p \in \mathbb{Q}\), \(\lvert x \rvert<1\), and
  \[\binom{p}{r}\coloneq \frac{p(p-1)\cdots(p-r+1)}{r!}.\]
\end{theorem}
\begin{corollary}{}{}
  Clearly,
  \[(a+x)^p=a^p\left(1+\frac{x}{a}\right)^p=a^p \sum_{r=0}^{\infty}{\binom{p}{r}\frac{x^r}{a^r}},\]
  under the same conditions.
\end{corollary}
\begin{fact}
  We can expand \((a+x)^p\) in descending powers of \(x\) by using \((a+x)^p=x^p\left(1+\frac{a}{x}\right)^p\).
\end{fact}
\begin{note}
  Sometimes computing a couple terms can be useful in finding a pattern. For example, to get the coefficient of \(x^k\) explicitly.
\end{note}
\newpage
\section{APGP}
\begin{stbox}{Basics}{}
    \begin{center}
    \begin{tabular}{|Sc|Sc|Sc|}
    \hline
    & AP & GP\\
    \hline
    \multirow{2}{*}{\(u_n\)} & \multicolumn{2}{Sc|}{\(u_n=S_n-S_{n-1}\)}\\
    \cline{2-3}
    & \(u_n=a+(n-1)d\) & \(u_n=ar^{n-1}\)\\
    \hline
    \(S_n\) & 
    \begin{tabular}{@{}Sc@{}} 
      \(S_n=\dfrac{n}{2}[2a+(n-1)d]\)\\
      \(=\dfrac{n}{2}(a+\ell)\hspace{0.7cm}\)
    \end{tabular} & 
    \begin{tabular}{@{}Sc@{}}
      \(S_n=\dfrac{a\left(1-r^n\right)}{1-r}\)\\
      \(\hphantom{S_n}=\dfrac{a\left(r^n-1\right)}{r-1}\)
    \end{tabular}\\
    \hline
    \(S_\infty\) & Diverges to \(\pm\infty\) iff \(a\neq 0\) or \(d\neq 0\) & \(S_\infty=\dfrac{a}{1-r}\) when \(\lvert r \rvert<1\)\\
    \hline
    Prove AP/GP & 
    \begin{minipage}{5.5cm}
      \begin{enumerate}[label=\Roman*]
        \item Show \(u_n-u_{n-1}\) is a constant.
        \item Show \(u_n=a+(n-1)d\) explicitly.
      \end{enumerate}
    \end{minipage} &
    \begin{minipage}{5.5cm}
      \begin{enumerate}[label=\Roman*]
        \item Show \(\dfrac{u_n}{u_{n-1}}\) is constant.
        \item Show \(u_n=ar^{n-1}\) explicitly
      \end{enumerate}
    \end{minipage}\\
    \hline 
    Mean & 
    \begin{tabular}{@{}Sc@{}}
      \(u_{n+1}=\dfrac{u_n+u_{n+2}}{2}\).\\
      (Arithmetic Mean)
    \end{tabular} &
    \begin{tabular}{@{}Sc@{}}
      \(\dfrac{u_{n+1}}{u_n}=\dfrac{u_{n+2}}{u_{n+1}}\)\\
      \(\hspace{0.7cm}u_{n+1}^2=u_{n} \cdot u_{n+2}\)\\
      (Geometric Mean)
    \end{tabular}\\
    \hline
  \end{tabular}
  \end{center}
\end{stbox}
\begin{IN}
  Applications: Write out a few terms in a table and observe the trend. (You can literally say ``By observing a trend, \ldots'')
\end{IN}
\begin{GCSkills}{}
  Table function
  \begin{enumerate}
    \item Enter eqn into GC.
    \item 2nd graph to show table
    \item 2nd tblset for setup options
  \end{enumerate}
\end{GCSkills}
\section{Summation}
\begin{fact}
  \begin{alignat*}{3}
    &\sum_{i=m}^{n}{f(i)+g(i)}\quad& &=\quad \sum_{i=m}^{n}{f(i)}+\sum_{i=m}^{n}{g(i)}\\
    &\sum_{i=m}^{n}{af(i)}& &=\quad a\sum_{i=m}^{n}{f(i)}\\
    &\sum_{i=m}^{n}{a}& &=\quad (n-m+1)a\text{, for any constant a}\\
    &\sum_{i=m}^{n}{f(i)}& &=\quad \sum_{i=1}^{n}{f(i)}-\sum_{i=1}^{m-1}{f(i)} 
  \end{alignat*}
\end{fact}
\begin{note}
  \begin{itemize}
    \item Look out for sums being AP and GPs.
    \item Results to be provided:
    \begin{align*}
      \sum_{i=1}^{n}{i^2}&=\frac{n}{6}(n+1)(2n+1)\\
      \sum_{i=1}^{n}{i^3}&=\frac{1}{4}n^2(n+1)^2
    \end{align*}
  \end{itemize}
\end{note}
\section{Method of Differences}
\begin{stbox}{General Information}{}
  \[\sum_{i=1}^{n}{u_i}=\sum_{r=1}^{n}{f(r)-f(r-1)}=f(n)-f(0).\]
  \begin{itemize}
    \item Explain convergence of a function \(h(x)=f(x)+g(x)\): As \(n \to \infty\), \(f(x) \to 0\) and \(g(x) \to 0\). Hence, \(h(x)\) converges to...
  \end{itemize}
\end{stbox}
\begin{GCSkills}{}
  Know how to generate sequences using the two methods:
  \begin{enumerate}
    \item Table method --- Present by showing two consecutive values of \(n\) so that the values of the sequence are of opposite signs. E.g.:
    \begin{center}
      \begin{tabular}{|Sc|Sc|}
        \hline
        \(n\) & \(S_n\)\\
        \hline
        182 & \(561.28<0\)\\
        \hline
        183 & \(-1935.91<0\)\\
        \hline
      \end{tabular}
    \end{center}
    \item 2nd stat seq (\& we can use operations on seq, e.g. sum)
  \end{enumerate}
\end{GCSkills}