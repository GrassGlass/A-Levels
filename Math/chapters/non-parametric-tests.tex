\chapter{Non-Parametric Tests}
\section{Sign Test}
\begin{stbox}{General Information}
  \begin{itemize}
    \item A \emph{sign test}.
    \begin{enumerate}
      \item Let \(m\) be the population median of \(D=\rule{1.5cm}{0.01mm}-\rule{1.5cm}{0.01mm}\).
      \item 
      \begin{tabular}{|ll|}
        \hline
        Test & \(H_0\colon m=m_0\)\\
        against &\(H_1\colon\) 
        \begin{enumerate*}[itemjoin={\quad}]
          \item \(m<m_0\),
          \item \(m\neq m_0\),\quad or
          \item \(m>m_0\),
        \end{enumerate*}\\
        \multicolumn{2}{|l|}{at the \(100\alpha\%\) significance level.}\\
        \hline
      \end{tabular}
      \item ~
      \begin{table}[H]
        \centering
        \begin{tabular}{|Sc|Sc|Sc|Sc|Sc|Sc|}
          \hline
          [label in context] & 1 & 2 & 3 & \(\cdots\) & \(N\)\\
          \hline 
          Sign & \(+\) & \(0\) & \(-\) & \(\cdots\) & \(+\)\\
          \hline
        \end{tabular}
        \caption{The signs of \(d_1,d_2,\dots,d_N\), for a sign test. Instead of \(1,2,\dots,N\) the labeling/column headers can differ in the given context. E.g. \(A,B,\dots,K\). Similarly, the signs here are mere examples; the \(i\)th sign cell should be filled with \(+\) (\(-\)) [0] if \(\operatorname{sgn}(d_i)=1\) (\(=-1\)) [\(=0\)].}
        \label{table:sign-test-working-table}
      \end{table}
      \item Let \(X_{+}\) be the number of \(`+'\). Under \(H_0\), \(X_{+}\sim\Binom(\hyperlink{non-parametric-tests-n-value}{n},1/2)\), \(x_{+}=\rule{0.5cm}{0.01mm}\). (Alternatively, \(X_{-}\) can also be used.)
      \item Since \(p\text{-value}=\rule{1cm}{0.01mm}<100\alpha\%\) (\(\geq 100\alpha\%\)), there is sufficient (insufficient) evidence, at the \(100\alpha\%\) significance level, to conclude that [\(H_1\) in context].
    \end{enumerate}
    \item \emph{Note.} The \(p\)-value for a sign test is given by
    \begin{table}[H]
      \centering
      \begin{tabular}{ScScScSc}
        \toprule
        \(H_1\) & \(m<m_0\) & \(m>m_0\) & \(m\neq m_0\)\\
        \midrule
        \(X_{+}\) & \(\Prob(X_{+}\leq x_{+})\) & \(\Prob(X_{+}\geq x_{+})\) & \(2\min\{\Prob(X_{+}\geq x_{+}),\Prob(X_{+}\leq x_{+})\}\)\\
        \midrule
        \(X_{-}\) & \(\Prob(X_{-}\geq x_{-})\) & \(\Prob(X_{-}\leq x_{-})\) & \(2\min\{\Prob(X_{-}\geq x_{-}),\Prob(X_{-}\leq x_{-})\}\)\\
        \bottomrule
      \end{tabular}
      \caption{The \(p\)-value for a sign test.}
      \label{table:sign-test-p-value}
    \end{table}
  \end{itemize}
\end{stbox}
\begin{note}
  Sign test. Suppose we have \(H_1\colon m\neq m_0\). To find the range of values of \(x_{+}\) that result in the rejection of \(H_0\), use GC to compute the following tables.
  \begin{table}[H]
    \centering
    \begin{tabular}{|Sc|Sc|}
      \hline
      \(x_{+}\) & \(\alpha/2-2\Prob(X_{+}\leq x_{+})\)\\
      \hline
      \(n-1\) & \(\rule{0.5cm}{0.01mm}>0\)\\
      \hline
      \(n\) & \(\rule{0.5cm}{0.01mm}>0\)\\
      \hline
      \(n+1\) & \(\rule{0.5cm}{0.01mm}<0\)\\
      \hline
    \end{tabular}\hspace{1cm}
    \begin{tabular}{|Sc|Sc|}
      \hline
      \(x_{+}\) & \(\alpha/2-2\Prob(X_{+}\geq x_{+})\)\\
      \hline
      \(N-1\) & \(\rule{0.5cm}{0.01mm}<0\)\\
      \hline
      \(N\) & \(\rule{0.5cm}{0.01mm}<0\)\\
      \hline
      \(N+1\) & \(\rule{0.5cm}{0.01mm}>0\)\\
      \hline
    \end{tabular} 
  \end{table}
  Then, we conclude that \(x_{+}\leq n\) or \(x_{+}\geq N\).
\end{note}
\section{Wilcoxon Matched-Pairs Signed Rank Test}
\begin{note}
  Assumptions needed for the Wilcoxon Matched-Pairs Signed Rank Test:
  \begin{enumerate}
    \item The data within each pair are dependent on each other, but pairs are independent of each other.
    \item The distribution of the differences is continuous and symmetrical.
  \end{enumerate}
\end{note}
\begin{stbox}{General Information}
  \begin{itemize}
    \item A Wilcoxon matched-pairs signed rank test. 
    \begin{enumerate}
      \item Let \(m\) be the population median of \(D=\rule{1.5cm}{0.01mm}-\rule{1.5cm}{0.01mm}\).
      \item 
      \begin{tabular}{|ll|}
        \hline
        Test & \(H_0\colon m=0\)\\
        against &\(H_1\colon\) 
        \begin{enumerate*}[itemjoin={\quad}]
          \item \(m<\highlight[yellow]{0}\),
          \item \(m\neq \highlight[yellow]{0}\),\quad or
          \item \(m>\highlight[yellow]{0}\),
        \end{enumerate*}\\
        \multicolumn{2}{|l|}{at the \(100\alpha\%\) significance level.}\\
        \hline
      \end{tabular}
      \item ~
      \begin{table}[H]
        \centering
        \begin{tabular}{|Sc|Sc|Sc|Sc|Sc|Sc|}
          \hline
          [label in context] & 1 & 2 & 3 & \(\cdots\) & \(N\)\\
          \hline 
          \(d\) & \(d_1\) & \(0\) & \(d_3\) & \(\cdots\) & \(d_N\)\\
          \hline
          Rank & \(1\) & \(0\) & \(5\) & \(\cdots\) & \(2\)\\
          \hline
        \end{tabular}
        \caption{The value of the differences \(d_1,d_2,\dots,d_N\), which are then ranked according to their absolute size \(\lvert d_i \rvert\). For our syllabus, each \(d_i\) is always distinct.
        % When the absolute difference is identical across two or more columns, \protect\hyperlink{tied-ranks-example}{take the average rank}.
        }
        \label{table:wilcoxon-working-table}
      \end{table}
    \end{enumerate}
    \begin{minipage}{0.6\textwidth}
      \begin{enumerate}
        \setcounter{enumi}{3}
        \item
        \begin{itemize}[label=\(\circ\)]
          \item \(t_{-}=\rule{0.5cm}{0.01mm}+\rule{0.5cm}{0.01mm}+\dots+\rule{0.5cm}{0.01mm}=\rule{0.5cm}{0.01mm}\)
          \item \(t_{+}=\rule{0.5cm}{0.01mm}+\rule{0.5cm}{0.01mm}+\dots+\rule{0.5cm}{0.01mm}=\rule{0.5cm}{0.01mm}\)
          \item The test statistic is \(T\coloneq\min\{T_{-},T_{+}\}=\rule{0.5cm}{0.01mm}\).
          \item Reject \(H_0\) if \(T=\rule{0.5cm}{0.01mm}\). (see table \ref{table:wilcoxon-critical-region})
        \end{itemize}
      \end{enumerate}
    \end{minipage}%
    \hspace{0.15cm}
    \begin{minipage}{0.3\textwidth}
      \begin{remark}
        \[\frac{\hyperlink{non-parametric-tests-n-value}{n}(\hyperlink{non-parametric-tests-n-value}{n}+1)}{2}=t_{-}+t_{+}.\]
      \end{remark}
      \vspace{0.7cm}
    \end{minipage}
    \begin{enumerate}
      \setcounter{enumi}{4}
      \item Since \(t=\rule{0.5cm}{0.01mm}\,\square\, \rule{0.5cm}{0.01mm}\), there is sufficient/insufficient evidence, at the \(100\alpha\%\) significance level, to conclude that [\(H_1\) in context].
    \end{enumerate}
    \item The test statistics \(T_{+}\) and \(T_{-}\) can also be used, depending on our preference.
      \item The critical regions for a Wilcoxon test, for each alternative hypothesis and test statistic \(T_{-}\) or \(T_{+}\). The value of \(c\) is obtained from MF26\(^{\hyperlink{non-parametric-tests-n-value}{*}}\). 
      
      \emph{Note.} the value of \(c\) may differ for a one-tail vs a two-tail test, so look at the table carefully, to obtain the correct value.
      \begin{table}[H]
        \centering
        \setlength{\tabcolsep}{12pt}
        \begin{tabular}{ScScScSc}
          \toprule
          \(H_1\) & \(m<0\) & \(m>0\) & \(m\neq 0\)\\
          \midrule
          \(T_{+}\) & \(T_{+}\leq c\) & \(T_{+}\geq \dfrac{n(n+1)}{2}-c\) & \(T_{+}\leq c\)\quad or\quad\(T_{+}\geq \dfrac{n(n+1)}{2}-c\)\\
          \midrule
          \(T_{-}\) & \(T_{-}\geq \dfrac{n(n+1)}{2}-c\) & \(T_{-}\leq c\) & \(T_{-}\leq c\)\quad or\quad\(T_{-}\geq \dfrac{n(n+1)}{2}-c\)\\
          \midrule
          \(T\) & \multicolumn{2}{Sc}{\(T\leq c^{\hyperlink{wilcoxson-T=min-note}{1}}\)} & \(T\leq c\)\quad or\quad\(T\geq \dfrac{n(n+1)}{2}-c\)\\
          \bottomrule
        \end{tabular}
        \caption{The critical regions for Wilcoxon tests.}
        \label{table:wilcoxon-critical-region}
      \end{table}
      \begin{footnotesize}
        {}\(^{\protect\hypertarget{wilcoxson-T=min-note}{1}}\)Assuming \(T_{-}\geq T_{+}\) for \(m<0\), and \(T_{+}\geq T_{-}\) for \(m>0\).
      \end{footnotesize}
      \setlength{\tabcolsep}{6pt}
      \item For large sample sizes \(n\geq 21\), we use the approximation 
      \[T\sim\Normal\left( \frac{n(n+1)}{4},\frac{n(n+1)(2n+1)}{24} \right)\]
      and conduct a one/two-tailed \(z\)-test.
  \end{itemize}
\end{stbox}
\begin{GCSkills}{}
  After calculating our list of differences \texttt{L\(_3\)}, we can calculate \(\texttt{L}_4=\lvert \texttt{L}_3 \rvert\) and use the G.C. to rank this list in ascending order:
  \begin{center}
    \texttt{stat} \(\Longrightarrow\) \texttt{2:SortA} \(\Longrightarrow\) \(\texttt{L}_4\).
  \end{center}
  This allows us to easily compute the ranks associated with each difference.
\end{GCSkills}
% \begin{note}
%   \hypertarget{tied-ranks-example}{}
%   Suppose wlog that \(0<\lvert d_1 \rvert<\lvert d_2 \rvert<\dots<\lvert d_p \rvert<\lvert d \rvert<\lvert d_{q+1} \rvert<d_{q+2}<\dots<d_n\) in the following table. Then, the rank assigned to the columns with identical absolute difference \(d\) is \(\highlight[black!15]{r\coloneq\frac{1}{q-p}\sum_{i=p+1}^{q}{i}}\):
%   % that \(\lvert d \rvert,\lvert d_1 \rvert,\lvert d_2 \rvert,\dots\) are unique and nonzero. Further assume that there is a one-to-one correspondence between \(\{r_1,r_2,\dots,r_p\}\cup\{r_{q+1},r_{q+2},\dots,r_n\}\) and \(\{1,2,\dots,s-1\}\cup\{s-p+q,s-p+q+1,\dots,n\}\). Then, for \(r\coloneq\frac{s+s+1+}{}\)
%   \begin{table}[H]
%     \centering
%     \begin{tabular}{|Sc|Sc|Sc|Sc|Sc|>{\columncolor[gray]{0.85}}Sc|>{\columncolor[gray]{0.85}}Sc|>{\columncolor[gray]{0.85}}Sc|>{\columncolor[gray]{0.85}}Sc|Sc|Sc|Sc|}
%       \hline
%       [label] & 1 & 2 & \(\cdots\) & \(p\) & \(p+1\) & \(p+2\) & \(\cdots\) & \(q\) & \(q+1\) & \(\cdots\) & \(n\)\\
%       \hline
%       \(D\) & \(d_1\) & \(d_2\) & \(\cdots\) & \(d_p\) & \(\pm d\) & \(\pm d\) & \(\cdots\) & \(\pm d\) & \(d_{q+1}\) & \(\cdots\) & \(d_n\)\\
%       \hline
%       Rank & \(r_1\) & \(r_2\) & \(\cdots\) & \(r_p\) & \(r\) & \(r\) & \(\cdots\) & \(r\) & \(r_{q+1}\) & \(\cdots\) & \(r_n\)\\
%       \hline
%     \end{tabular}
%     \caption{Identical absolute differences}
%     \label{table:tied-ranks-example}
%   \end{table}
% \end{note}
\begin{note}\hypertarget{non-parametric-tests-n-value}{}
  The value of \(n\) for the test statistic/MF26 critical region in both tests should be the number of columns with nonzero difference \(d\). i.e.
  \[n\coloneq\#\{i \,\vert\, d_i\neq 0\}=\#\text{cols}-\#\{i \,\vert\, d_i=0\}.\]
\end{note}
\begin{note}
  If we need to use both the sign test and a Wilcoxon test on the same sample, then consider creating just a single table, as shown below. 
  \begin{table}[H]
    \centering
    \begin{tabular}{|Sc|Sc|Sc|Sc|Sc|Sc|}
      \hline
      [label in context] & 1 & 2 & 3 & \(\cdots\) & \(n\)\\
      \hline 
      \(d\) & \(d_1\) & \(0\) & \(d_3\) & \(\cdots\) & \(d_n\)\\
      \hline
      Sign & \(+\) & \(0\) & \(-\) & \(\cdots\) & \(+\)\\
      \hline
      Rank & \(1\) & \(0\) & \(5\) & \(\cdots\) & \(2\)\\
      \hline
    \end{tabular}
    \caption{Combined table for both the sign test and Wilcoxon test.}
    \label{table:sign-Wilcoxon-combined}
  \end{table}
\end{note}
\begin{note}
  How do you improve the Wilcoxon test used in [the previous part]?
  \begin{center}
    Increase the sample size for the test.
  \end{center}
\end{note}
\begin{note}
  State the circumstances under which a non-parametric test would be used rather than a parametric test.
  \begin{center}
    \parbox{0.9\textwidth}{
      We use a non-parametric test, rather than a parametric test, when:
      \begin{enumerate}
        \item The population is not known to be normally distributed.
        \item The population mean is not the best way to measure central tendency.
        \item The measurement scale has no predetermined rank or ordering. 
      \end{enumerate}
    }
  \end{center}
\end{note}
\begin{note}
  \hypertarget{wilcoxon>t}{}
  Why is it not appropriate to use a paired-sample \(t\)-test? 
  \begin{center}
    \parbox{0.9\textwidth}{
      There is no contextual evidence to support the assumption that \(D_1,D_2,\dots,D_n\) are normally distributed. So, conducting a paired-sample \(t\)-test may result in unreliable results, given our small sample size \(n\). 
    }
  \end{center}
\end{note}
\begin{note}
  State the precautions that should be taken to avoid (statistical) bias. 
  \begin{center}
    \parbox{0.9\textwidth}{
      Choose any approperate ones.
      \begin{enumerate}
        \item The test should be \emph{`blind'}. [Testers in context] should not know which of the [two variations involved in the test, in context] they are [tasting/wearing/etc, in context]. If the [testers] knew, their preconceptions may affect \rule{2cm}{0.01mm}.
        \item Pick a random sample of \(n\) [testers].
        \item The \emph{order} of the test --- whether the [first variation] or [second variation] comes first --- should be randomised.
        \item The [testers] should not communicate with each other.
        \item There should be sufficient rest time between the two runs, so that the running timing of the second run would not be affected due to fatigue.
      \end{enumerate}
    }
  \end{center}
\end{note}
\begin{note}
  \hypertarget{wilcoxson>sign}{}
  Explain why it is better to conduct a \textcolor{green!70!black}{Wilcoxon} test than a \textcolor{red}{sign} test.
  \begin{center}
    \parbox{0.9\textwidth}{
     While a sign test only considers the sign of the differences, a Wilcoxon test takes into account both the sign and \emph{magnitude} of the differences. Therefore, a Wilcoxon test is more reliable, as it incorporates more (relevant) information about the data.
    }
  \end{center}
  % \begin{center}
  %   \parbox{0.9\textwidth}{
      
  %   }
  % \end{center}
\end{note}
% \begin{note}
%   Explain why it might be better to conduct a sign test, rather than a Wilcoxon test. 
%   \begin{center}
%     \parbox{0.9\textwidth}{
      
%     }
%   \end{center}
% \end{note}
\begin{note}
  Explain why it is appropriate use a Wilcoxon test in this situation.
  \begin{enumerate}
    \item[\textcolor{green!70!black}{\checkmark}] Explain why a Wilcoxon test is better than other tests (\hyperlink{wilcoxon>t}{\(t\)-test}, \hyperlink{wilcoxson>sign}{sign test}).
    \item[\textcolor{red}{\(\times\)}] Explain why the assumptions of the Wilcoxon test hold true, and hence, the test can be carried out to reach a reliable conclusion. 
  \end{enumerate}
\end{note}
\begin{note}
  Explain why a sign test is more suitable/a \textcolor{red}{Wilcoxon} test is inappropriate.
  \begin{center}
    \parbox{0.9\textwidth}{
      Choose any approperiate ones
      \begin{enumerate}
        \item The data here is non-numeric and is not measured on an ordinal scale. Hence, it is inappropriate to conduct a Wilcoxon test. A sign test is better, as the data can still be represented by positive and negative responses --- denoting \rule{1cm}{0.01mm} and \rule{1cm}{0.01mm}, respectively.
        \item The magnitude of the differences is irrelevant because \rule{1cm}{0.01mm}. So, a sign test --- which only accounts for the sign of the differences --- is more appropriate.
        \item In this case, the data has too many \emph{tied ranks}. Thus, the conclusion obtained from a Wilcoxon test may not be reliable.
        \item An additional assumption that the distribution of the differences \(D=\rule{0.5cm}{0.01mm}-\rule{0.5cm}{0.01mm}\) must be continuous and symmetric about the median. 
      \end{enumerate}
    }
  \end{center}
\end{note}
\begin{example}{A tricker question, involving an unknown in the data provided.}{}
  Let \(m\) be the median of \(D\colon X-Y\). For the data in Table \ref{table:wilcoxon-unknown-variable-raw-data}, assume that there are no tied ranks, and \(x_i\neq y_i\) for each \(1\leq i\leq 7\). Carry out a Wilcoxon test, at the \(5\%\) significance level, to determine if the data supports the alternative hypothesis \(H_1\colon m>0\).
  \begin{table}[H]
    \centering
    \begin{tabular}{ScScScScScScScSc}
      \toprule
      Index & \(1\) & \(2\) & \(3\) & \(4\) & \(5\) & \(6\) & \(7\)\\
      \midrule
      \(x_i\) & \(4\) & \(8\) & \(7\) & \(7\) & \(1\) & \(9\) & \(9\)\\
      \(y_i\) & \(6\) & \(9\) & \(3\) & \(4\) & \(a\) & \(1\) & \(2\)\\
      \bottomrule
    \end{tabular}
    \caption{Data with an unknown variable \(a\in {\mathbb{Z}^{+}}\).}
    \label{table:wilcoxon-unknown-variable-raw-data}
  \end{table}
  \rule{20cm-137.0549pt}{0.05mm}
  First, we calculate the differences. Since \(x_i\neq y_i\), we have \(a\neq 1\). In fact, \(a\neq 1,2,3,4,7,8\) because \(d_i\neq d_j\), for \(i\neq j\). Thus, \(a=6,7\) or \(a\geq 10\). The corresponding rank \(r_5\) is hence \(5\) or \(7\). 
  \begin{table}[H]
    \centering
    \begin{tabular}{Sc>{\columncolor[gray]{0.85}}Sc>{\columncolor[gray]{0.85}}ScScSc>{\columncolor[gray]{0.85}}ScScSc}
      \toprule
      Index & \(1\) & \(2\) & \(3\) & \(4\) & \(5\) & \(6\) & \(7\)\\
      \midrule
      \(d_i\) & \(-2\) & \(-1\) & \(4\) & \(3\) & \(1-a\) & \(8\) & \(7\)\\
      \(\lvert d_i \rvert\) & \(2\) & \(1\) & \(4\) & \(3\) & \(a-1\) & \(8\) & \(7\)\\
      rank \(r_i\) & \(2\) & \(1\) & \(4\) & \(3\) & \(r_5\) & \(r_6\) & \(r_7\)\\
      \bottomrule
    \end{tabular}
    \caption{The values of the differences \(d_i\) and the associated ranks. The columns highlighted in grey are those with negative differences \(d_i\).}
    \label{table:wilcoxon-unknown-variable-proceessed-data-with-calculated-differences}
  \end{table}
  Now,
  \[t_{-}=2+1+r_5=8,10 \qquad\text{and}\qquad t_{+}=7(7+1)/2-t_{-}=25-r_5=20,18.\]
  Hence, the test statistic \(T\coloneq\min\{T_{-},T_{+}\}=T_{-}\), where we reject \(H_0\) if \(T\leq 3\). So, since \(t_{-}=3+r_5>3\), we do not reject \(H_0\). 
\end{example}