\chapter{Vectors}
\begin{note}
  A useful fact about cross products. For any \(a_i,b_i\in \mathbb{R}\):
  \[\left\lVert 
    \begin{pmatrix}
      a_1\\
      a_2\\
      a_3
    \end{pmatrix}\times
    \begin{pmatrix}
      b_1\\
      b_2\\
      b_3
    \end{pmatrix}
   \right\rVert=
  \begin{vmatrix}
    1 & a_1 & b_1\\
    1 & a_2 & b_2\\
    1 & a_3 & b_3\\
  \end{vmatrix}.\]
\end{note}
\begin{longtable}{|Sc|Sc|}
  \hline
  Lines & Planes\\
  \hline
  \multicolumn{2}{|Sc|}{Equivalent Forms}\\
  \hline
  \begin{minipage}{0.4\linewidth}
    \begin{enumerate}
      \item Vector Equation: \[\ell \colon \mathbf{r}=\mathbf{a}+\lambda \mathbf{m}\text{, }\lambda \in \mathbb{R},\]
      \item Cartesian Equation: 
      \[\frac{x-a_1}{m_1}=\frac{y-a_2}{m_2}=\frac{z-a_3}{m_3.}\]
    \end{enumerate}
\end{minipage} & 
\begin{minipage}{0.4\linewidth}
\begin{enumerate}
  \item Vector Equation: 
  \[\Pi \colon \mathbf{r}=\mathbf{a}+\lambda
  \mathbf{m_1}+\mu \mathbf{m_2},\ \lambda,\mu\in\mathbb{R}.\]
  \item Scalar Product Form: 
  \[\Pi \colon \mathbf{r} \cdot \mathbf{n}=p\]
  where the scalar \(p\coloneq  \mathbf{a}\cdot \mathbf{n}\),
  \item Cartesian Equation:
  \[n_1x+n_2x+n_3z=p\]
  where the normal vector\\
  \(\mathbf{n}\coloneq \begin{pmatrix}
    n_1 & n_2 & n_3
  \end{pmatrix}^\top\).
\end{enumerate}
\end{minipage}\\
\hline
\multicolumn{2}{|Sc|}{Foot of Perpendicular}\\
\hline
\begin{minipage}{0.4\linewidth}
  \begin{enumerate}
    \item[M1:] 
    \begin{enumerate}
      \item \(\overrightarrow{ON}=\mathbf{a}+\lambda \mathbf{m}\),
      \item \(\overrightarrow{QN} \cdot \mathbf{m}=0\), solve for \(\lambda\),
      \item Substitute \(\lambda\) into (a).
    \end{enumerate}
    \item[M2:] 
    \begin{enumerate}
      \item \(\overrightarrow{AN}=\left(\overrightarrow{AQ} \cdot \mathbf{\widehat{m}}\right)\mathbf{\widehat{m}}\),
      \item \(\overrightarrow{ON}=\overrightarrow{OA}+\overrightarrow{AN}\).
    \end{enumerate}
  \end{enumerate}
\end{minipage} & 
\begin{minipage}{0.4\linewidth}
  \begin{enumerate}
    \item[M1:]
    \begin{enumerate}[label=(\alph*)]
      \item \(\overrightarrow{ON}=\overrightarrow{OQ}+\lambda\mathbf{n}\),
      \item \(\overrightarrow{ON}\cdot\mathbf{n}=p\), solve for \(\lambda\),
      \item  Substitute \(\lambda\) into (a).
    \end{enumerate}
    \item[M2:]
    \begin{enumerate}[label=(\alph*)]
      \item \(\overrightarrow{QN}=\left( \overrightarrow{QA}\cdot\widehat{\mathbf{n}} \right)\widehat{\mathbf{n}}\),
      \item \(\overrightarrow{ON}=\overrightarrow{OQ}+\overrightarrow{QN}\).
    \end{enumerate}
  \end{enumerate}
\end{minipage}\\
\hline
\newpage
\hline
\multicolumn{2}{|Sc|}{Shortest Distance of Point To Line, \(QN\)}\\
\hline
\begin{minipage}{0.4\linewidth}
  \begin{enumerate}
    \item[M1:] \(\norm{\overrightarrow{AQ} \times \mathbf{\widehat{m}}}\).
    \item[M2:]
    \begin{enumerate}
      \item \(AN= \norm{\overrightarrow{AQ}\cdot \mathbf{\widehat{m}}} \),
      \item Pythagoras' Theorem.
    \end{enumerate}
    \item[M3:] Using the foot of perpendicular, find distance \(QN\).
  \end{enumerate}
\end{minipage} &
\begin{minipage}{0.4\linewidth}
  \begin{enumerate}
    \item[M1:] \(\norm{\overrightarrow{AQ}\cdot \mathbf{\widehat{n}}} \).
    \item[M2:] Distance of plane to \emph{origin}: 
    
    If \(\Pi \colon \mathbf{r}\cdot \mathbf{n}=p\), then \(\dfrac{p}{\norm{\mathbf{n}}}\) is the shortest distance from the origin to the plane \(\Pi\). 
    
    \emph{Note:}
    \begin{itemize}
      \item If \(\dfrac{p}{\norm{\mathbf{n}}}>0\), then \(\Pi\) is `above' the origin.
      \item If \(\dfrac{p}{\norm{\mathbf{n}}}<0\), then \(\Pi\) is `below' the origin.
    \end{itemize}
    \item[M3:] Using the foot of perpendicular, then find distance \(QN\).
  \end{enumerate}
\end{minipage}\\
\hline
The Relationship Between Two Lines & The Relationship Between Two Planes\\
\hline
\begin{minipage}{0.4\linewidth}
  \begin{enumerate}
    \item Parallel, Non-Intersecting
    \begin{enumerate}
      \item \(\mathbf{m_1}\newparallel \mathbf{m_2}\),
      \item Solving \(\mathbf{r_1}=\mathbf{r_2}\) gives no real solution. Or, show that \(\mathbf{a_1}\) does not lie in \(\ell_2\).
    \end{enumerate}
    \item Parallel, Coinciding
    \begin{enumerate}
      \item \(\mathbf{m_1}\newparallel \mathbf{m_2}\),
      \item \(\mathbf{a}\) lies in \(\ell_1\) and \(\ell_2\).
    \end{enumerate}
    \item Non-Parallel, Intersecting
    \begin{enumerate}
      \item \(\mathbf{m_1}\) not \(\newparallel \mathbf{m_2}\),
      \item Solve \(\mathbf{r_1}=\mathbf{r_2}\) to find intersection.
    \end{enumerate} 
    \item Skew Lines\\
    (Non-Parallel, Non-Intersecting)
    \begin{enumerate}
      \item \(\mathbf{m_1}\) not \(\newparallel \mathbf{m_2}\),
      \item Solving \(\mathbf{r_1}=\mathbf{r_2}\) gives no real solution. 
    \end{enumerate}
  \end{enumerate}
\end{minipage} &
\begin{minipage}{0.4\linewidth}
  \begin{enumerate}
    \item Distinct Parallel Planes: 
    \begin{enumerate}
      \item Show that \(\mathbf{n_1}\newparallel\mathbf{n_2}\),
      \item Find a vector \(\mathbf{b}\) for which
      \begin{enumerate}[label=(\roman*)]
        \item \(\mathbf{b}\cdot \mathbf{n_1}= p_1\),
        \item \(\mathbf{b}\cdot \mathbf{n_2}\neq p_2\).
      \end{enumerate}
    \end{enumerate}
    \item Same Plane:
    \begin{enumerate}
      \item Show that \(\mathbf{n_1}\newparallel\mathbf{n_2}\),
      \item Find a vector \(\mathbf{b}\) for which
      \begin{enumerate}[label=(\roman*)]
        \item \(\mathbf{b}\cdot \mathbf{n_1}= p_1\),
        \item \(\mathbf{b}\cdot \mathbf{n_2}= p_2\).
      \end{enumerate}
    \end{enumerate}
    \item Intersect in a line \(\ell\); To find this line:
    \begin{enumerate}[label=M\arabic*:]
      \item \(\mathbf{n_1}\times \mathbf{n_2}\) gives the direction vector. So find a common point with simultaneous equations.
      \item Solving system of linear equations, from the \emph{cartesian} form of the planes, using G.C.
    \end{enumerate}
  \end{enumerate}
\end{minipage}\\
\hline
\newpage
\hline
\multicolumn{2}{|Sc|}{The Relationship Between A Line and A Plane}\\
\hline 
\multicolumn{2}{|Sc|}{
\begin{minipage}{0.8\linewidth}
  \begin{enumerate}
    \item \(\ell\) lies in \(\Pi\)
    \begin{enumerate}[label=M\arabic*:]
      \item
      \begin{enumerate}
        \item Show \(\mathbf{m}\cdot \mathbf{n}=0\) so \(\ell \newparallel \Pi\).
        \item Then \(\mathbf{a}_\ell\cdot \mathbf{n}=p\) tells us \(\ell\) lies in \(\Pi\).
      \end{enumerate}
      \item Substitute \(\ell\) into \(\Pi\) and show the system (of lin eqns) is consistent for all \(\lambda\).
    \end{enumerate}
    \item \(\ell \newparallel \Pi\) but nonintersecting
    \begin{enumerate}[label=M\arabic*:]
      \item 
      \begin{enumerate}
        \item Show \(\mathbf{m}\cdot \mathbf{n}=0\) so \(\ell \newparallel \Pi\).
        \item Then \(\mathbf{a}_\ell\cdot \mathbf{n} \neq p\) tells us \(\ell\) and \(\Pi\) are nonintersecting.
      \end{enumerate}
      \item Substitute \(\ell\) into \(\Pi\), and show the system (of lin eqns) is inconsistent.
    \end{enumerate} 
    \item Intersect at one point
    \begin{enumerate}
      \item Check that \(\mathbf{m}\cdot \mathbf{n} \neq 0\).
      \item Then, to find the point of intersection of the plane \(\Pi\colon \mathbf{r}\cdot \mathbf{n}=p\) with the line \(\ell\colon \mathbf{r}=\mathbf{a}+\lambda \mathbf{m}\),
      we solve for \(\lambda\) using simultaneous equations or G.C.
    \end{enumerate}
  \end{enumerate}
\end{minipage}}\\
\hline
\multicolumn{2}{|Sc|}{The Point of Reflection}\\
\hline
\multicolumn{2}{|Sc|}{  
\begin{minipage}{0.8\linewidth}
\begin{center}
  \begin{enumerate}
    \item Find foot of perpendicular \(\overrightarrow{ON}\)
    \item Notice \(\overrightarrow{OA'}=\overrightarrow{OA}+2\overrightarrow{AN}=2\overrightarrow{ON}-\overrightarrow{OA}\).
  \end{enumerate}
\end{center}
\end{minipage}}\\
\hline
  \multicolumn{2}{|Sc|}{Angle Between}\\
  \hline
  \multicolumn{2}{|Sc|}{
    \begin{tabular}{Sc|Sc|Sc}
      \begin{minipage}{0.267\linewidth}
        \centering
        Two Lines
      \end{minipage} &
      \begin{minipage}{0.267\linewidth}
        \centering
        Line and Plane
      \end{minipage} &
      \begin{minipage}{0.267\linewidth}
        \centering
        Two Planes
      \end{minipage}\\
    \end{tabular}}\\
    \hline
    \multicolumn{2}{|Sc|}{
    \begin{tabular}{Sc|Sc|Sc}
      \begin{minipage}{0.267\linewidth}\vspace{-0.5cm}
        \[\theta=\cos^{-1}\left\lvert \mathbf{\widehat{m}_1}\cdot {\mathbf{\widehat{m}_2}}\right\rvert.\]
      \end{minipage} &
      \begin{minipage}{0.267\linewidth}\vspace{-0.5cm}
        \[\theta=\sin^{-1}\left\lvert \mathbf{\widehat{m}}\cdot \mathbf{\widehat{n}} \right\rvert .\]
      \end{minipage} &
      \begin{minipage}{0.267\linewidth}\vspace{-0.5cm}
        \[\theta=\cos^{-1}\left\lvert \mathbf{\widehat{n}_1}\cdot \mathbf{\widehat{n}_2} \right\rvert.\]
      \end{minipage}\\
    \end{tabular}}\\
    \hline
\end{longtable}
\begin{note}
  Describe the line \(\ell\colon\mathbf{r}=\mathbf{a}+\lambda\mathbf{m},\lambda\in \mathbb{R}\) geometrically.
  \begin{center}
    \parbox{0.9\textwidth}{
      It is the line \emph{passing through the fixed point} with position vector \(\mathbf{a}\) and is \emph{parallel to} \(\mathbf{m}\). 
    }
  \end{center}
\end{note}
\begin{note}
  Let \(\pi\) be a plane containing the line \(\ell\) and \(P\) be a point. The foot of perpendicular of \(P\) on \(\pi\) is not necessarily the same as the foot of perpendicular of \(P\) on \(\ell\). 
\end{note}
\begin{note}
  When finding the plane \(\pi\) that is `above' another plane \(\Pi\colon \mathbf{r}\cdot \mathbf{n}=p\) by a constant \(q\) units, take note of the direction of \(\mathbf{n}=
  \begin{pmatrix}
    n_1 & n_2 & n_3
  \end{pmatrix}^{\top}\). If \(n_3>0\), then \(\mathbf{n}\) is pointing `upwards'. So, \(\pi\colon \mathbf{r}\cdot\widehat{\mathbf{n}}=\highlight[green!50]{+}q\). Otherwise \(n_3<0\), meaning \(\mathbf{n}\) is pointing `downwards'. Hence, \(\pi\colon \mathbf{r}\cdot\widehat{\mathbf{n}}=\highlight[green!50]{-}q\).
  \footnotetext[0]{I'm slightly skeptical that Cambridge will not make clear which side is considered `upwards/downwards'. But, just in case, this is a good point to note.}
\end{note}