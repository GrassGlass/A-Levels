\documentclass[oneside]{book}

\usepackage{../Style/Diagrams}
\usepackage{../Style/Master}
\usepackage{../Style/boxes}
\usepackage{../Style/DefNoteFact}
\usepackage{../Style/QnsProof}
\usepackage{../Style/Thms}
\usepackage{../Style/Env}
\usepackage{../Style/NewCommands}

\usepackage{fancyhdr}
\pagestyle{fancy}
\fancyhead[l]{Shao Hong}
\fancyhead[c]{FMA Blended Learning}
\fancyhead[r]{\today}
\fancyfoot[c]{\thepage}
\renewcommand{\headrulewidth}{0.2pt} %Creates a horizontal line underneath the header
\setlength{\headheight}{15pt} %Sets enough space for the header
\begin{document}
\begin{Qns}{}{}
    Let \(P(x)=\sum_{i=0}^{n}{a_ix^i}\) be a polynomial of even degree \(n\), such that \(a_n\) is positive and \(a_0\) is negative. Prove that \(P(x)\) has at least one positive real root and one negative real root.
\end{Qns}
\begin{proof}
    Recall that \(\lim_{x \to \pm \infty}1/x=0\). By limit laws, \(\lim_{x \to \pm \infty}\sum_{i=0}^{n}{a_ix^{i-n}}=0\). Therefore, there exists \(M>0\) so \(a_n>\sum_{i=0}^{n-1}{-a_iM^{i-n}}\). That is, \(P(M)>0\). Since \(P(0)<0\) and all polynomials are continuous, there is a positive root of \(P(y)=0\) by the Intermediate Value Theorem. Similarly there is also a negative root.
\end{proof}
\end{document}