\documentclass[tikz]{standalone}
\usepackage{pgfplots} %http://www.ctan.org/pkg/pgfplots
\pgfplotsset{compat=1.8, set layers=standard}

\begin{document}

\pgfplotsset{
    integral axis/.style={
        axis lines=middle,
        axis line style = very thick,
        enlarge y limits=upper,
        axis equal image, width=12cm,
        xlabel=\Large$x$, ylabel=\Large$y$,
        ytick=\empty,
        xticklabel style={font=\Large, text height=1.5ex, anchor=north},
        samples=100
    },
    integral/.style={
            domain=2:10,
            samples=4
    },
    integral fill/.style={
            integral,
            draw=none, fill=#1,
            on layer=axis background
        },
        integral fill/.default=cyan!10,
        integral line/.style={
            integral,
            line width=0.5mm,
            draw=#1
        },
        integral line/.default=black
}

\begin{tikzpicture}[
    % The function that is used for all the plots
    declare function={f=((0.1*x)^3*e^(0.2*x));},
]
\begin{axis}[
    integral axis,
    ymin=0,
    xmin=0.75, xmax=11.25,
    domain=1.5:10.5,
    xtick={2,14/3,22/3,10},
    xticklabels={\(a=x_0\), \(x_1\), \(x_2\), \(b=x_3\)},
]
% The function
\addplot [line width=0.5mm, cyan!75!blue,domain=0:10.1] {f} node [anchor=south,xshift=-2mm] {\Large$y=f(x)$};

% The filled area under the approximate integral
\addplot [integral fill=cyan!15] {f} \closedcycle;

% The approximate integral
\addplot [integral line=black] {f};

% The vertical lines between the segments
\addplot [integral, ycomb] {f};

% The highlighted segment
% \addplot [integral fill=cyan!35, domain=6:7, samples=2] {f} \closedcycle;
\end{axis}
\end{tikzpicture}
\end{document}