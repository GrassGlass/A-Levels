%* Source: https://tex.stackexchange.com/a/305225 (ellipse)
% https://tex.stackexchange.com/a/599565 (tangent)
% https://tex.stackexchange.com/a/253010 (bisector command)
\documentclass[tikz,border=1mm]{standalone}
\usepackage{tkz-euclide, calc}
\usetikzlibrary{decorations.pathreplacing, decorations.markings, angles}
\NewDocumentCommand\bisector{O{}mmmO{10pt}}{
  \path[#1] let
    \p1 = ($(#3)!1cm!(#2)$),
    \p2 = ($(#3)!1cm!(#4)$),
    \p3 = ($(\p1) + (\p2) - (#3)$)
    in    
    ($(#3)!#5!($(#3)!1cm!(\p3)$)$) -- ($($(#3)!1cm!(\p3)$)!#5!(#3)$) ;
}
\begin{document}
\tikzset{-<-/.style={decoration={
  markings,
  mark=at position #1 with {\arrow[scale=2]{Classical TikZ Rightarrow[reversed]}}},postaction={decorate}}}
\begin{tikzpicture}[dot/.style={draw,fill,circle,inner sep=1pt}]
  \def\a{4} % large half axis
  \def\b{3} % small half axis
  \def\angle{55} % angle at which X is placed
  % Draw the ellipse
  \draw (0,0) ellipse ({\a} and {\b});
  % Draw the inner lines and labels
  \draw[dashed] (-\a,0) coordinate (A)
    -- (\a,0) coordinate (B);
  \draw[dashed] (0,-\b) coordinate (D)
    -- (0,\b) coordinate (C);
  \coordinate[label={below left:\(O\)}] (O) at (0,0);
  % Nodes at the focal points
  \node[dot,label={below:$F_1$}] (F1) at ({-sqrt(\a*\a-\b*\b)},0) {};
  \node[dot,label={below:$F_2$}] (F2) at ({+sqrt(\a*\a-\b*\b)},0) {};
  % Node on the rim, connected to foci
  \node[dot,label={\angle:\(P\)}] (X) at (\angle:{\a} and {\b}) {};
  \draw[-<-=.5] (F1) -- (X);
  \draw[-<-=.5] (X) -- (F2);
  % Directrices
  \draw (-16/5,-\b) -- (-16/5,\b);
  \draw (16/5,-\b) -- (16/5,\b);
  \coordinate (D1) at (-16/5,-\b);
  \coordinate (D2) at (16/5,-\b);
  % Tangent and angle
  \tkzDefLine[bisector out](F1,X,F2) \tkzGetPoint{K}
  \draw[blue] ($(X)!-0.5!(K)$) coordinate (U) --($(X)!+0.5!(K)$) coordinate (V)
    pic [draw,blue,line width=0.25mm,double,angle radius=0.5cm] {angle = U--X--F1}
    pic [draw,blue,line width=0.25mm,double,angle radius=0.5cm] {angle = F2--X--V};
  % \bisector[draw,blue]{U}{X}{F1}
  % \bisector[draw,blue]{F2}{X}{V}
  % Brace
%   \draw[decorate,decoration=brace,draw=red] (C) -- (O);
\end{tikzpicture}
\end{document}