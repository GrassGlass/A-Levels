\documentclass{article}
\usepackage{tikz}
\usetikzlibrary{math}
\usepackage[active,tightpage]{preview}
\PreviewEnvironment{tikzpicture}
\setlength\PreviewBorder{1pt}
%
% File name: Dottie-point.tex
% Description: 
% A representation of the Dottie Number is shown.
% 
% Date of creation: March, 4th, 2023.
% Date of last modification: March, 4th, 2023.
% Author: Efraín Soto Apolinar.
% https://www.aprendematematicas.org.mx/author/efrain-soto-apolinar/instructing-courses/
% Source: page 136 of the book titled:
% Glossarium Mathematica (English version) by Efraín Soto Apolinar.
% (To be published soon)
%
% Terms of use:
% According to the license: Attribution-NonCommercial-ShareAlike 4.0 International (CC BY-NC-SA 4.0) 
% https://creativecommons.org/licenses/by-nc-sa/4.0/
% Your commitment to the terms of use is greatly appreciated.
%
\begin{document}
%
\begin{tikzpicture}[scale=2.0]
		\tikzmath{function f(\x) {return cos(\x r);};}
		\pgfmathsetmacro{\xdottie}{0.739085}
		\pgfmathsetmacro{\ydottie}{f(\xdottie)}
		%
		\pgfmathsetmacro{\dx}{0.5}
		\pgfmathsetmacro{\xi}{-0.5*pi-0.25*\dx}
		\pgfmathsetmacro{\xf}{0.5*pi+0.5*\dx}
		\pgfmathsetmacro{\ejex}{0.5*pi+0.5*\dx} % x
		\pgfmathsetmacro{\ejey}{0.5*pi-0.5*\dx}
		% The line:  $y = x$
		\draw[blue] (0,0) -- (\ejey,\ejey) node [very near end,sloped,above] {\footnotesize$y = x$};
		% Origin
		\node[below left] at (0,0) {$O$};
		% Coordinate axis
		\draw[thick,->] (\xi,0) -- (\ejex,0) node [right]{$x$};
		\foreach \x/\xtext in {-1/-1,1/1,-1.5708/-\frac{\pi}{2},1.5708/\frac{\pi}{2}}{
			\draw[thick] (\x,1pt) -- (\x,-1pt) node [below] {$\xtext$};
		}
		\draw[thick,->] (0,-0.25) -- (0,\ejey) node [above]{$y$};
		\draw[thick] (1pt,1) -- (-1pt,1) node [left] {$1$};
		% The graph of the cosine function
		\draw[blue,thick,->] plot[domain=\xi:\xf,smooth] (\x,{cos(\x r)}) 
			node[below]{\footnotesize$y = \cos(x)$};
		% Initial values for the fixed-point method
		\pgfmathsetmacro{\xa}{-1.0}
		\pgfmathsetmacro{\ya}{f(\xa)}
		\draw[red] (\xa,0) -- (\xa,\ya); 
		% Cycle for the fixed-point method
		\foreach \i [remember=\x as \xlast (initially \xa), remember=\y as \ylast (initially \ya)] in {1,2,...,15}{
			\tikzmath{\x = \ylast; \y=f(\ylast); }
			\draw[red] (\xlast,\ylast) -- (\x,\ylast) -- (\x,\y);
		}
		% Indication of the location of the Dottie point:
		\fill[red] (\xdottie,\ydottie) circle (0.25pt);
		%
	\end{tikzpicture}
%
\end{document}