\documentclass[border=5,varwidth,tikz]{standalone}
\usepackage[utf8]{inputenc}
\usepackage[T1]{fontenc}
\usepackage{nicefrac}
\usepackage{amsmath}

\usepackage{tikz}
%\usetikzlibrary{calc}
% https://tex.stackexchange.com/a/466846
\usetikzlibrary{backgrounds}
\usepackage{nicefrac}
\newcommand\half{\hbox{$\nicefrac12$}}
\newcommand\onehalf{\hbox{$\nicefrac32$}}
\begin{document}

\foreach \m in {0,1,...,359}
{
\pgfmathsetmacro\p{\m/180*pi}
%
\begin{tikzpicture}

\def\xs{2.}
\def\xc{\xs}
\def\yc{-3.}

\begin{scope}[shift={(0, 0)}]

\begin{scope}[semithick,->,>=stealth]
\draw (0, -1.3) -- (0, 1.3);
\draw (-1.3, 0) -- (1.3, 0);
\end{scope}
\begin{scope}[font=\tiny, inner sep=2pt]
\draw (-0.05, 1.) -- node [above left] {$1$} node [above right,orange] {$\half\pi$} coordinate (Y1) (0.05, 1.);
\draw (-0.05, -1.) -- node [below left] {$-1$} node [below right,orange] {$\onehalf\pi$} coordinate (YM1) (0.05, -1.);
\draw (1, -0.05) -- node [below right] {$1$} node [above right,orange] {$0 / 2\pi$} coordinate (X1) (1, 0.05) ;
\draw (-1, -0.05) -- node [below left] {$-1$} node [above left,orange] {$\pi$} coordinate (XM1) (-1, 0.05);
\end{scope}

\begin{scope}[blue!50!green]
\draw (0, 0) circle (1);
\end{scope}

\end{scope}

\begin{scope}[shift={(0, {\yc})}]
\draw (-1.2, -1.2) -- (1.2, 1.2);
\end{scope}

\begin{scope}[shift={({\xs}, 0)}]

\begin{scope}[semithick,->,>=stealth]
\draw (0, -1.3) -- (0, 1.3);
\draw (-0.04*pi, 0) -- (2.05*pi, 0);
\end{scope}
\begin{scope}[font=\tiny]
\coordinate (S0) at (0,0); 
\draw (-0.05, 1.)    -- node [above left] {$1$}    coordinate (S1)     (0.05, 1.);
\draw (-0.05, -1.)   -- node [below left] {$-1$}  coordinate (SM1)   (0.05, -1.);
\draw (pi/2, 0.05)   -- node [below] {$\displaystyle\half\pi$}    +(0.0, -0.1);
\draw (pi, 0.05)     -- node [below] {$\pi$}      +(0.0, -0.1);
\draw (pi*3/2, 0.05) -- node [below] {$\onehalf\pi$}   +(0.0, -0.1);
\draw (2*pi, 0.05)   -- node [below] {$2\pi$}     +(0.0, -0.1);
\end{scope}

\begin{scope}[blue]
\draw (0, 0) 
 \foreach \n in {0,...,100}  { 
    -- ({\n*(0.02*pi)}, {sin(\n*0.02*pi r)})
};
\end{scope}

\end{scope}

\begin{scope}[shift={({\xc}, {\yc})}]

\begin{scope}[semithick,->,>=stealth]
\draw (0, -1.3) -- (0, 1.3);
\draw (-0.04*pi, 0) -- (2.05*pi, 0);
\end{scope}
\begin{scope}[font=\tiny]
\coordinate (C0) at (0,0);
\draw (-0.05, 1.)    -- node [above left] {$1$} coordinate (C1)   (0.05, 1.);
\draw (-0.05, -1.)   -- node [below left] {$-1$} coordinate (CM1)  (0.05, -1.);
\draw (pi/2, 0.05)   -- node [below] {$\half\pi$}    +(0.0, -0.1);
\draw (pi, 0.05)     -- node [below] {$\pi$}      +(0.0, -0.1);
\draw (pi*3/2, 0.05) -- node [below] {$\onehalf\pi$}   +(0.0, -0.1);
\draw (2*pi, 0.05)   -- node [below] {$2\pi$}     +(0.0, -0.1);
\end{scope}

\begin{scope}[green]
\draw (0, 1) 
 \foreach \n in {0,...,100}  { 
    -- ({\n*(0.02*pi)}, {cos(\n*0.02*pi r)})
};
\end{scope}
\end{scope}

\begin{scope}[on background layer={gray, help lines}]
  \draw (X1) |- (C1) -- +(2.08*pi, 0);
  \draw (XM1) |- (CM1)  -- +(2.08*pi, 0);
  \draw (Y1) -- (S1) -- +(2.08*pi, 0);
  \draw (YM1) -- (SM1)  -- +(2.08*pi, 0);
  \draw (YM1) |- (C0);
  \draw (XM1) -- (S0);
\end{scope}

\begin{scope}
\begin{scope}[on background layer]
 \fill[orange!50!yellow] (0,0) -- (.5,0) arc [rotate=-90, start angle=0, end angle={\p r}, radius=.5];
\end{scope}
 \coordinate (P) at ({\p r}:1);
 \coordinate (PM) at (-{\p r}:1);
 \coordinate (PR) at ({(pi-(\p)) r}:1);
 \coordinate (S) at ({\xs+\p}, {sin(\p r)});
 \coordinate (C) at ({\xc+\p}, {\yc+cos(\p r)});
 \draw [->,>=stealth,thick,red,line cap=round] (0,0) -- %node [pos=1, auto=right, font=\scriptsize] {$e^{j\phi}$} 
(P);
\iffalse
 \draw [->,>=stealth,black!40] (0,0) -- %node [pos=1, auto=right, font=\scriptsize] {$e^{-j\phi}$}
 (PM) ;
 \draw [->,>=stealth,black!20] (0,0) -- %node [pos=1, auto=right, font=\scriptsize] {$-e^{-j\phi}$} 
(PR) ;

 \draw [thin,black!30] (P) -- (PM);
 \draw [thin,black!10] (P) -- (PR);
\fi
\draw [blue!50!black] (P) -- (S) node [fill, circle, inner sep=.5pt] {};
\draw [green!50!black] (P) |- (C) node [fill, circle, inner sep=.5pt] {};;
\end{scope}

\end{tikzpicture}
}

\end{document}