% Author: Izaak Neutelings (June 2020)
% Inspiration:
%  https://courses.physics.ucsd.edu/2011/Summer/session1/physics2c/diffraction.pdf
%  https://tex.stackexchange.com/questions/201830/periodic-shading-in-tikz
\documentclass[border=3pt,tikz]{standalone}
\usepackage[outline]{contour} % glow around text
\usepackage{physics}
\usepackage{xcolor}
\usepackage{etoolbox} %ifthen
\usetikzlibrary{calc}
\usetikzlibrary{arrows,arrows.meta}
\usetikzlibrary{decorations.markings}
\usetikzlibrary{angles,quotes} % for pic (angle labels)
\usetikzlibrary{fadings}
\tikzset{>=latex} % for LaTeX arrow head
\contourlength{1.4pt}

\colorlet{wall}{blue!30!black}
\colorlet{myblue}{blue!70!black}
\colorlet{myred}{red!70!black}
\colorlet{mydarkred}{red!50!black}
\colorlet{mylightgreen}{green!60!black!70}
\colorlet{mygreen}{green!60!black}
\colorlet{myredgrey}{red!50!black!80}
\colorlet{myshadow}{blue!30!black!90}
\tikzstyle{wave}=[myblue,thick]
\tikzstyle{mydashed}=[black!70,dashed,thin]
\tikzstyle{mymeas}=[{Latex[length=3,width=2]}-{Latex[length=3,width=2]},thin]
\tikzstyle{mysmallarr}=[-{Latex[length=3,width=2]}]


\newcommand\rightAngle[4]{
  \pgfmathanglebetweenpoints{\pgfpointanchor{#2}{center}}{\pgfpointanchor{#3}{center}}
  \coordinate (tmpRA) at ($(#2)+(\pgfmathresult+45:#4)$);
  \draw[white,line width=0.6] ($(#2)!(tmpRA)!(#1)$) -- (tmpRA) -- ($(#2)!(tmpRA)!(#3)$);
  \draw[mydarkred] ($(#2)!(tmpRA)!(#1)$) -- (tmpRA) -- ($(#2)!(tmpRA)!(#3)$);
}
\newcommand\lineend[2]{
  \def\w{0.1} \def\c{30}
  \draw[mygreen] (#1)++(#2:\w) to[out=#2-180-\c,in=#2+\c] (#1)
                               to[out=#2+\c-180,in=#2-\c]++ (#2-180:\w);
}
\def\tick#1#2{\draw[thick] (#1) ++ (#2:0.1) --++ (#2-180:0.2)}

% INTERFERENCE FADING
\begin{tikzfadingfrompicture}[name=interference]
  \def\lambd{0.5} % wavelength
  \foreach \r in {1,...,15}
    \foreach \j in {1,...,25}
       \path [line width=\lambd*\j,draw=transparent!0,opacity=0.04]
          (0,0) circle (\lambd*\r); %(0:\r) arc (0:180:\r);
\end{tikzfadingfrompicture}

\tikzset{
  declare function={
    int_arg(\y,\lambd,\a,\L) = \a*\y/sqrt(\L*\L+\y*\y)/\lambd; %sin(\x);
    int_one(\y,\lambd,\a,\L) = (sin(180*int_arg(\y,\lambd,\a,\L))/(pi*int_arg(\y,\lambd,\a,\L)))^2;
    int_two(\y,\lambd,\a,\L) = cos(180*int_arg(\y,\lambd,\a,\L))^2;
    int_arg_ang(\t,\lambd,\a) = \a*sin(\t)/\lambd;
    int_one_ang(\t,\lambd,\a) = (sin(180*int_arg_ang(\t,\lambd,\a))/(pi*int_arg_ang(\t,\lambd,\a)))^2;
    int_two_ang(\t,\lambd,\a) = cos(180*int_arg_ang(\t,\lambd,\a))^2;
  }
}
\usepackage{newpxtext, eulerpx}

\begin{document}


% % INTERFERENCE FADING
% \begin{tikzpicture}
%   \def\a{2}  % distance sources
%   \def\W{8} % distance between walls
%   \def\H{8}  % total wall height
%   %\clip (-\W/2,0) rectangle ++(\W,\H);
%   \path[fill=myshadow,path fading=interference,fit fading=false,fading transform={shift={(0,\a/2)}}] %,shift={(-2,0)}
%     (0,-\H/2) rectangle ++(\W,\H);
%   \path[fill=myshadow,path fading=interference,fit fading=false,fading transform={shift={(0,-\a/2)}}] %rotate=45
%     (0,-\H/2) rectangle ++(\W,\H);
%   %\draw (0,0) --++ (0,\H);
% \end{tikzpicture}


% TWO SPLIT
\begin{tikzpicture}[
    nodal/.style={mylightgreen,dashed,very thin},
    declare function={
      %xnode(\n,\dn,\lam,\f) = sqrt( (\n^2+(\n+\dn)^2)*\lambd^2/2 - (\n^2-(\n+\dn)^2)^2*\lambd^4/(4*\a^2) - \a^2/4 );
      xnode(\n,\dn,\lam,\f) = \lam/\f*sqrt( \n^2*(\f^2-\dn^2)+\n*\dn*(\f^2-\dn^2)+\dn^2*\f^2/2-(\f^4+\dn^4)/4 );
      ynode(\n,\dn,\lam,\a) = (2*\n*\dn+\dn^2)*\lam/(2*\f);
      % intensity(\y,\lam,\a,\L) = int_one(\y,\lambd,\a,\L)*int_two(\y,\lambd,\a,\L);
      intensity(\y,\lam,\a,\L) = cos(180*\a*\y/(2*\lam*sqrt(\L*\L+\y*\y)))^2;
    }
  ]
  \def\A{1.7}
  \def\d{3.4}     % slit distance
  \def\a{1.2}     % slit size
  \def\lambd{0.2} % wavelength
  \def\k{15}      % x -> theta
  \def\xmax{2}
  \def\D{2.3*\xmax}
  \def\L{4.0}
% 
  \def\H{5.4}       % total wall height
  \def\h{2.8}       % plane wave height
  \def\t{0.15}      % wall thickness
  \def\N{21}        % number of waves
  \def\lambd{0.20}  % wavelength
  \def\R{\N*\lambd} % wave radius
  \def\Nlines{3}    % number of nodal lines
  %\def\r{0.06}      % point source radius
  %\def\nmax{10}
  \def\nsamples{100}
  \def\ang{62}
  
  \begin{scope}
    \clip (-\t/2,-\H/2) rectangle (\L,\H/2);
    %\clip (-\t/2,0.7*\a) -- (0.6*\L,\H/2) -- (\L,\H/2) --
    %      (\L,-\H/2) -- (0.6*\L,-\H/2) -- (-\t/2,-0.7*\a) -- cycle;
    
    % NODAL LINES
    \draw[nodal]
      (0.08*\N*\lambd,0) -- (1.06*\R,0);
    \coordinate (NP0) at (\L,0);  % to avoid "Dimension too large error"
    \foreach \dn [evaluate={
                   \f=\a/\lambd;
                   \nmin=2.5+0.2*\dn; %0.501*(-\dn+\f)
                   \nmax=10; %(NP0)
                   \c=int(\dn<\f);
                   \y=\L/sqrt((\a/(\lambd*\dn))^2-1);
                 }] in {1,...,\Nlines}{
      \coordinate (NP+\dn) at (\L,\y);  % to avoid "Dimension too large error"
      \coordinate (NP-\dn) at (\L,-\y); % to avoid "Dimension too large error"
      \ifnum\c=1
        \draw[nodal,variable=\n,samples=\nsamples,smooth]
          plot[domain=\nmin:\nmax] ({xnode(\n,\dn,\lambd,\f)},{ynode(\n,\dn,\lambd,\f)})
          -- (NP+\dn);
        \draw[nodal,variable=\n,samples=\nsamples,smooth]
          plot[domain=\nmin:\nmax] ({xnode(\n,\dn,\lambd,\f)},{-ynode(\n,\dn,\lambd,\f)})
          -- (NP-\dn);
      \fi
    }
    
    % WAVES
    \foreach \i [evaluate={\R=\i*\lambd;}] in {1,...,\N}{
      \ifodd\i
        \draw[myblue,line width=0.8] (0,\a/2)++(\ang:\R) arc (\ang:-\ang:\R);
        \draw[myred,line width=0.8] (0,-\a/2)++(\ang:\R) arc (\ang:-\ang:\R);
      \else
        \draw[myblue!80,line width=0.1] (0,\a/2)++(\ang:\R) arc (\ang:-\ang:\R);
        \draw[myred!80,line width=0.1] (0,-\a/2)++(\ang:\R) arc (\ang:-\ang:\R);
      \fi
    }
  \end{scope}
  
  % PLANE WAVES
  \foreach \i [evaluate={\x=-\i*\lambd;}] in {0,...,5}{
    \ifodd\i
      \draw[myblue,line width=0.8] (\x,-\h/2) -- (\x,\h/2);
    \else
      \draw[myblue,line width=0.1] (\x,-\h/2) -- (\x,\h/2);
    \fi
  }
  
  % WALL
  \fill[wall]
    (\t/2,\a/2-\d/2) rectangle (-\t/2,-\a/2+\d/2)
    (\t/2,\a/2+\d/2) rectangle (-\t/2,\H/2)
    (\t/2,-\a/2-\d/2) rectangle (-\t/2,-\H/2)
    (\L,-\H/2) rectangle (\L+\t,\H/2);
  
  % SHADES
  \begin{scope}[shift={(1.08*\L,0)}]
    \def\yz{\L/sqrt((\a/\lambd)^2-1)} % m = +- 1/2
    \def\yZ{\L/sqrt((\a/\lambd/2)^2-1)} % m = +- 1
    \clip (0,-\H/2) rectangle (1.1*\A,\H/2);
    \fill[white] (0,-\H/2) rectangle++ (\A,\H); % to fill seams
    \foreach \i [evaluate={\n=0.5*\i;\yn=\L/sqrt((\a/(2*\lambd*\n))^2-1);
                 }] in {1,...,\Nlines}{
      \ifodd\i % if even
        \fill[myshadow] (0,{-\yn-0.1}) rectangle++ (\A,0.2); % to fill seams
        \fill[myshadow] (0,{ \yn-0.1}) rectangle++ (\A,0.2); % to fill seams
      \fi
    }
    \path[left color=myshadow,right color=myshadow,middle color=white,shading angle={180}]
      (0,{-\yz}) rectangle (\A,{\yz});
    \foreach \i [evaluate={
                  \n=0.5*\i;
                  \m=0.5*(\i+1);
                  \yn=\L/sqrt((\a/(2*\lambd*\n))^2-1);
                  \ym=\L/sqrt((\a/(2*\lambd*\m))^2-1);
                  \dang=mod(\i,2)*180;
                 }] in {1,...,\Nlines}{
      \path[left color=myshadow,right color=white,shading angle={\dang}]
        (0,\yn) rectangle (\A,\ym);
      \path[left color=myshadow,right color=white,shading angle={180+\dang}]
        (0,-\yn) rectangle (\A,-\ym);
    }
  \end{scope}
  
  % INTENSITY
  \begin{scope}[shift={(1.1*\L+1.1*\A,0)}]
    \draw[->,thick] (-0.08*\A,0) -- (1.3*\A,0) node[right=-2] {\(I\)}; % I axis
    \draw[->,thick] (0,-0.52*\H) -- (0,0.54*\H) node[right] {\(y\)}; % y axis
    \draw[nodal] (NP0) --++ (0.15*\L+2.1*\A,0); % green nodal lines
    \foreach \i [evaluate={\y=\L/sqrt((\a/(\lambd*\i))^2-1)}] in {1,...,\Nlines}{ % green nodal lines
      \draw[nodal] (NP+\i) --++ ({0.15*\L+1.1*\A+\A*(1/(abs(\y)^2+1))^(0.5)*intensity(\y,\lambd,\a,\L)},0);
      \draw[nodal] (NP-\i) --++ ({0.15*\L+1.1*\A+\A*(1/(abs(\y)^2+1))^(0.5)*intensity(\y,\lambd,\a,\L)},0);
    }
    % \draw[wave,dashed,thin,variable=\t,samples=\nsamples,smooth,domain=-\H/2:\H/2]
    % plot({\A*int_one_ang(\t,\lambd,\a)},\t);
    % \draw[wave,purple,variable=\t,samples=2*\nsamples,smooth,domain=-\H/2:\H/2]
    % plot({\A*int_one_ang(\t,\lambd,\a)*int_two_ang(\t,\lambd,\d)},\t);
    % 
    % \draw[myred,thick,variable=\y,samples=\nsamples,smooth,domain=-\H/2:\H/2]
    % plot({(1/(abs(\y)+1))^(0.8)*\A*intensity(\y,\lambd,\a,\L)},\y);
    \draw[blue!50!red,thick,variable=\y,samples=\nsamples,smooth,domain=-\H/2:\H/2]
    plot({(1/(abs(\y)^2+1))^(0.5)*\A*intensity(\y,\lambd,\a,\L)},\y);
  \end{scope}
  
\end{tikzpicture}
\begin{tikzpicture}
    \message{Two slits^^J}
    \def\A{1.7}
    \def\d{3.4}     % slit distance
    \def\a{1.2}     % slit size
    \def\lambd{0.2} % wavelength
    \def\k{15}      % x -> theta
    \def\xmax{2}
    \def\D{2.3*\xmax}
    \def\L{4.0}
    \def\nsamples{80}
    
    
    % WAVE 3
    \begin{scope}[shift={(2*\D,0)}]
      \draw[->,thick,black]
        (-1.05*\xmax,0) -- (1.12*\xmax,0);
      \draw[wave,dashed,thin,variable=\t,samples=\nsamples,smooth,domain=-\xmax:\xmax]
        plot(\t,{\A*int_one_ang(\k*\t,\lambd,\a)});
      \draw[wave,purple,variable=\t,samples=2*\nsamples,smooth,domain=-\xmax:\xmax]
        plot(\t,{\A*int_one_ang(\k*\t,\lambd,\a)*int_two_ang(\k*\t,\lambd,\d)});
    \end{scope}
    
  \end{tikzpicture}

\end{document}