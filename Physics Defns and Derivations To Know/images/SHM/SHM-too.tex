\documentclass[tikz]{standalone}
\usepackage{tikz,pgfplots}
\usepackage{newpxtext, eulerpx}
\usetikzlibrary{fpu}
\begin{document}
\begin{tikzpicture}[>=latex,
    rednode/.style={rectangle, draw=red, minimum height=0.1pt, minimum width=10pt, inner sep=0pt,},
    bluenode/.style={rectangle, draw=blue, minimum height=0.1pt, minimum width=10pt, inner sep=0pt,},
    violetnode/.style={rectangle, draw=violet, minimum height=0.1pt, minimum width=10pt, inner sep=0pt,},
    nonode/.style={rectangle, draw=red, minimum height=0pt, minimum width=0pt, inner sep=0pt,},]
    \begin{axis}[
      axis x line=center,
      axis y line=center,
      xtick={-1,1},
      ytick={1},
      every tick/.style={},
      xticklabels={\(-x_0\),\(x_0\)},
      yticklabels={\(\frac{1}{2}m\omega^2x_0^2\)},
      yticklabel style={anchor=south east},
    %   xticklabel=\empty,
    %   yticklabel=\empty,
    %   ticks=none,
      xlabel={\(x\)},
      ylabel={\(E\)},
      xlabel style={right},
      ylabel style={above},
      xmin=-1.1,
      xmax=1.1,
      ymin=-0.1,
      ymax=1.1,
      ]
    %   Origin
    \node[below left] at (axis cs:0,0) {\(0\)};
    % https://tex.stackexchange.com/questions/33607/easy-curves-in-tikz
    % K.E.
    \addplot[domain=-1:1,blue,samples=100] {-x^2+1};
    % P.E.
    \addplot[domain=-1:1,violet,samples=100] {x^2};
    % Dotted lines
    \addplot[domain=-1:1,black,dashed] {1};
    \draw[black,dashed] (axis cs:-1,0) to (axis cs:-1,1);
    \draw[black,dashed] (axis cs:1,0) to (axis cs:1,1);
    \end{axis}
    \matrix [draw,below left] at (current bounding box.south east) {
        \node [bluenode,label=right:\textcolor{blue}{Kinetic energy}] {};\\
        \node [violetnode,label=right:\textcolor{violet}{Potential energy}] {}; \\
        };
\end{tikzpicture}
\begin{tikzpicture}[>=latex,
    rednode/.style={rectangle, draw=red, minimum height=0.1pt, minimum width=10pt, inner sep=0pt,},
    bluenode/.style={rectangle, draw=blue, minimum height=0.1pt, minimum width=10pt, inner sep=0pt,},
    violetnode/.style={rectangle, draw=violet, minimum height=0.1pt, minimum width=10pt, inner sep=0pt,},
    nonode/.style={rectangle, draw=red, minimum height=0pt, minimum width=0pt, inner sep=0pt,},]
    \begin{axis}[
      axis x line=center,
      axis y line=center,
      ytick={1},
      yticklabels={\(\omega x_0\)},
      xticklabel style={anchor=north east},
      xtick={1},
      xticklabel style={anchor=north west},
      xticklabels={\(x_0\)},
      every tick/.style={},
      yticklabel style={anchor=south east},
      extra y ticks={-1},
      extra y tick style={
        yticklabel={\(-\omega x_0\)},
        yticklabel style={anchor=north east}},
      extra x ticks={-1},
      extra x tick style={
          xticklabel={\(-x_0\)},
          xticklabel style={anchor=north east}},
    %   xticklabel=\empty,
    %   yticklabel=\empty,
    %   ticks=none,
      xlabel={\(x\)},
      ylabel={\(v\)},
      xlabel style={right},
      ylabel style={above},
      xmin=-1.1,
      xmax=1.1,
      ymin=-1.1,
      ymax=1.1,
      axis equal,
      ]
    %   Origin
    \node[below left] at (axis cs:0,0) {\(0\)};
    % https://tex.stackexchange.com/questions/33607/easy-curves-in-tikz
    % Circle
    \draw (axis cs:0,0) circle (2.59cm);
    \end{axis}
\end{tikzpicture}
\begin{tikzpicture}[>=latex,
  rednode/.style={rectangle, draw=red, minimum height=0.1pt, minimum width=10pt, inner sep=0pt,},
  bluenode/.style={rectangle, draw=blue, minimum height=0.1pt, minimum width=10pt, inner sep=0pt,},
  violetnode/.style={rectangle, draw=violet, minimum height=0.1pt, minimum width=10pt, inner sep=0pt,},
  nonode/.style={rectangle, draw=red, minimum height=0pt, minimum width=0pt, inner sep=0pt,},]
  \begin{axis}[
    axis x line=center,
    axis y line=center,
    xtick={-1,1},
    ytick={-1,1},
    every tick/.style={},
    xticklabels={\(-x_0\),\(x_0\)},
    yticklabels={\(-\omega^2x_0\),\(\omega^2x_0\)},
  %   xticklabel=\empty,
  %   yticklabel=\empty,
  %   ticks=none,
    xlabel={\(x\)},
    ylabel={\(a\)},
    xlabel style={right},
    ylabel style={above},
    xmin=-1.1,
    xmax=1.1,
    ymin=-1.1,
    ymax=1.1,
    ]
  %   Origin
  \node[below left] at (axis cs:0,0) {\(0\)};
  % https://tex.stackexchange.com/questions/33607/easy-curves-in-tikz
  % acceleration a(x)
  \addplot[domain=-1:1,blue,samples=100] {-x};
  % Dotted lines
  \addplot[domain=-1:-0.35,black,dashed] {1};
  \addplot[domain=0:1,black,dashed] {-1};
  \draw[black,dashed] (axis cs:-1,0) to (axis cs:-1,1);
  \draw[black,dashed] (axis cs:1,-0.2) to (axis cs:1,-1);
  \end{axis}
\end{tikzpicture}
\begin{tikzpicture}[>=latex,
  rednode/.style={rectangle, draw=red, minimum height=0.1pt, minimum width=10pt, inner sep=0pt,},
  bluenode/.style={rectangle, draw=blue, minimum height=0.1pt, minimum width=10pt, inner sep=0pt,},
  violetnode/.style={rectangle, draw=violet, minimum height=0.1pt, minimum width=10pt, inner sep=0pt,},
  nonode/.style={rectangle, draw=red, minimum height=0pt, minimum width=0pt, inner sep=0pt,},]
  \begin{axis}[
    axis x line=center,
    axis y line=center,
    xtick={1/8,2/8,...,1},
    ytick={1},
    every tick/.style={},
    xticklabels={\(\frac{T}{8}\),\(\frac{T}{4}\),\(\frac{3T}{8}\),\(\frac{T}{2}\),\(\frac{5T}{8}\),\(\frac{3T}{4}\),\(\frac{7T}{8}\),\(T\)},
    yticklabels={\(E_T\)},
  %   xticklabel=\empty,
  %   yticklabel=\empty,
  %   ticks=none,
    xlabel={\(x\)},
    ylabel={\(E\)},
    xlabel style={right},
    ylabel style={above},
    xmin=-0.07,
    xmax=1.1,
    ymin=-0.1,
    ymax=1.1,
    ]
  %   Origin
  \node[below left] at (axis cs:0,0) {\(0\)};
  % https://tex.stackexchange.com/questions/33607/easy-curves-in-tikz
  % K.E.
  \addplot[domain=0:1,blue,samples=100] {(cos(deg(2*pi*x)))^2};
  % P.E.
  \addplot[domain=0:1,violet,samples=100] {(sin(deg(2*pi*x)))^2};
  % Dotted lines
  \addplot[domain=0:1,black,dashed] {1};
  % \draw[black,dashed] (axis cs:1/8,0) to (axis cs:1/8,1);
  \draw[black,dashed] (axis cs:1/4,0) to (axis cs:1/4,1);
  % \draw[black,dashed] (axis cs:3/8,0) to (axis cs:3/8,1);
  \draw[black,dashed] (axis cs:1/2,0) to (axis cs:1/2,1);
  % \draw[black,dashed] (axis cs:5/8,0) to (axis cs:5/8,1);
  \draw[black,dashed] (axis cs:3/4,0) to (axis cs:3/4,1);
  % \draw[black,dashed] (axis cs:7/8,0) to (axis cs:7/8,1);
  \draw[black,dashed] (axis cs:1,0) to (axis cs:1,1);
  \end{axis}
  \matrix [draw,below left] at (current bounding box.south east) {
    \node [bluenode,label=right:\textcolor{blue}{Kinetic energy}] {};\\
    \node [violetnode,label=right:\textcolor{violet}{Potential energy}] {}; \\
    };
\end{tikzpicture}
\begin{tikzpicture}[>=latex,
  rednode/.style={rectangle, draw=red, minimum height=0.1pt, minimum width=10pt, inner sep=0pt,},
  bluenode/.style={rectangle, draw=blue, minimum height=0.1pt, minimum width=10pt, inner sep=0pt,},
  violetnode/.style={rectangle, draw=violet, minimum height=0.1pt, minimum width=10pt, inner sep=0pt,},
  nonode/.style={rectangle, draw=red, minimum height=0pt, minimum width=0pt, inner sep=0pt,},]
  \begin{axis}[
    axis x line=center,
    axis y line=center,
    xtick={-1,1},
    ytick={1},
    every tick/.style={},
    xticklabels={},
    yticklabels={\(E_T\)},
  %   xticklabel=\empty,
  %   yticklabel=\empty,
  %   ticks=none,
    xlabel={\(x\)},
    ylabel={\(E\)},
    xlabel style={right},
    ylabel style={above},
    xmin=-0.1,
    xmax=2.6,
    ymin=-0.1,
    ymax=7.1,
    ]
  %   Origin
  \node[below left] at (axis cs:0,0) {\(0\)};
  % https://tex.stackexchange.com/questions/33607/easy-curves-in-tikz
  % No damping
  \addplot[domain=0:2.5,black,samples=500] {1/(sqrt((1-x^2)^2)+(2*x*0)^2)};
  % Light damping
  \addplot[domain=0:2.5,yellow!85!black,samples=200,smooth] {1/(sqrt((1-x^2)^2)+(2*x*0.22)^2)};
  % Heavy damping
  \addplot[domain=0:2.5,orange,smooth,samples=200] {1/(sqrt((1-x^2)^2)+(2*x*0.3)^2)};
  \addplot[domain=0:2.5,black,smooth,samples=500,dashed] {1/(sqrt(1-x^4))};
  % Dotted lines
  % \addplot[domain=0:1,black,dashed] {1};
  \draw[black,dashed] (axis cs:1,0) to (axis cs:1,1);
  \end{axis}
  \matrix [draw,below left] at (current bounding box.south east) {
    \node [bluenode,label=right:\textcolor{blue}{Kinetic energy}] {};\\
    \node [violetnode,label=right:\textcolor{violet}{Potential energy}] {}; \\
    };
\end{tikzpicture}
\end{document}