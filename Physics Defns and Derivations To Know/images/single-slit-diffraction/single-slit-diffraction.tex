% Author: Izaak Neutelings (June 2020)
% Inspiration:
%  https://courses.physics.ucsd.edu/2011/Summer/session1/physics2c/diffraction.pdf
%  https://tex.stackexchange.com/questions/201830/periodic-shading-in-tikz
\documentclass[border=3pt,tikz]{standalone}
\usepackage[outline]{contour} % glow around text
\usepackage{physics}
\usepackage{xcolor}
\usepackage{etoolbox} %ifthen
\usetikzlibrary{calc}
\usetikzlibrary{arrows,arrows.meta}
\usetikzlibrary{decorations.markings}
\usetikzlibrary{angles,quotes} % for pic (angle labels)
\usetikzlibrary{fadings}
\tikzset{>=latex} % for LaTeX arrow head
\contourlength{1.6pt}

\colorlet{wall}{blue!30!black}
\colorlet{myblue}{blue!70!black}
\colorlet{myred}{red!65!black}
\colorlet{mypurple}{red!50!blue!95!black!75}
\colorlet{myshadow}{blue!30!black!90}
\colorlet{mydarkred}{red!50!black}
\colorlet{mylightgreen}{green!60!black!70}
\colorlet{mygreen}{green!60!black}
\colorlet{myredgrey}{red!50!black!80}
\tikzstyle{wave}=[myblue,thick]
\tikzstyle{mydashed}=[black!70,dashed,thin]
\tikzstyle{mymeas}=[{Latex[length=3,width=2]}-{Latex[length=3,width=2]},thin]
\tikzstyle{mysmallarr}=[-{Latex[length=3,width=2]}]

\tikzset{
  declare function={
    int_arg(\y,\lam,\a,\L) = \a*\y/sqrt(\L*\L+\y*\y)/\lam; %sin(\x);
    int_one(\y,\lam,\a,\L) = (sin(180*int_arg(\y,\lam,\a,\L))/(pi*int_arg(\y,\lam,\a,\L)))^2;
    int_two(\y,\lam,\a,\L) = cos(180*int_arg(\y,\lam,\a,\L))^2;
    int_arg_ang(\t,\lam,\a) = \a*sin(\t)/\lam;
    int_one_ang(\t,\lam,\a) = (sin(180*int_arg_ang(\t,\lam,\a))/(pi*int_arg_ang(\t,\lam,\a)))^2;
    int_two_ang(\t,\lam,\a) = cos(180*int_arg_ang(\t,\lam,\a))^2;
  }
}

\newcommand\rightAngle[4]{
  \pgfmathanglebetweenpoints{\pgfpointanchor{#2}{center}}{\pgfpointanchor{#3}{center}}
  \coordinate (tmpRA) at ($(#2)+(\pgfmathresult+45:#4)$);
  \draw[white,line width=0.6] ($(#2)!(tmpRA)!(#1)$) -- (tmpRA) -- ($(#2)!(tmpRA)!(#3)$);
  \draw[mydarkred] ($(#2)!(tmpRA)!(#1)$) -- (tmpRA) -- ($(#2)!(tmpRA)!(#3)$);
}
\newcommand\lineend[2]{
  \def\w{0.1} \def\c{30}
  \draw[mygreen] (#1)++(#2:\w) to[out=#2-180-\c,in=#2+\c] (#1)
                               to[out=#2+\c-180,in=#2-\c]++ (#2-180:\w);
}
\def\tick#1#2{\draw[thick] (#1) ++ (#2:0.1) --++ (#2-180:0.2)}

\usepackage{newpxtext, eulerpx}

\begin{document}


%% TWO SPLIT: interference x diffraction
%\begin{tikzpicture}
%  \def\A{3.5}
%  \def\d{3.4}     % slit distance
%  \def\a{0.9}     % slit size
%  \def\lambd{0.2} % wavelength
%  \def\k{15}      % x -> theta
%  \def\xmax{4.6}
%  \def\ymax{\A}
%  \def\N{4}
%  \def\D{2.3*\xmax}
%  \def\L{4.0}
%  \def\nsamples{80}
%  
%  \draw[->,thick,black] (-1.05*\xmax,0) -- (1.1*\xmax,0) node[below] {$\theta$};
%  \draw[->,thick,black] (0,-0.1*\A) -- (0,1.12*\A);
%  \draw[wave,variable=\t,samples=\nsamples,smooth,domain=-\xmax:\xmax]
%    plot(\t,{\A*int_one_ang(\k*\t,\lambd,\a)});
%  
%  \foreach \i [evaluate={\t=asin(\i*\lambd/\a)/\k;}] in {1,...,\N}{
%    \tick{-\t,0}{90} node[below=-1,scale=0.7] {$m=-\i$};
%    \tick{\t,0}{90} node[below=-1,scale=0.7] {$m=+\i$};
%  }
%  
%\end{tikzpicture}


% TWO SLIT: interference x diffraction
\begin{tikzpicture}
  \message{Two slits^^J}
  \def\A{1.7}
  \def\d{3.4}     % slit distance
  \def\a{1.2}     % slit size
  \def\lambd{0.2} % wavelength
  \def\k{15}      % x -> theta
  \def\xmax{2}
  \def\D{2.3*\xmax}
  \def\L{4.0}
  \def\nsamples{80}
  
  % WAVE 1
  \draw[->,thick,black]
    (-1.05*\xmax,0) -- (1.12*\xmax,0);
  \draw[wave,variable=\t,samples=\nsamples,smooth,domain=-\xmax:\xmax]
    plot(\t,{\A*int_one_ang(\k*\t,\lambd,\a)});
  \node at (\D/2,0.45*\A) {$\times$};
  
  % WAVE 2
  \begin{scope}[shift={(\D,0)}]
    \draw[->,thick,black]
      (-1.05*\xmax,0) -- (1.12*\xmax,0);
    \draw[wave,myred,variable=\t,samples=3*\nsamples,smooth,domain=-\xmax:\xmax]
      plot(\t,{\A*int_two_ang(\k*\t,\lambd,\d)});
    \node at (\D/2,0.45*\A) {$=$};
  \end{scope}
  
  % WAVE 3
  \begin{scope}[shift={(2*\D,0)}]
    \draw[->,thick,black]
      (-1.05*\xmax,0) -- (1.12*\xmax,0);
    \draw[wave,dashed,thin,variable=\t,samples=\nsamples,smooth,domain=-\xmax:\xmax]
      plot(\t,{\A*int_one_ang(\k*\t,\lambd,\a)});
    \draw[wave,mypurple,variable=\t,samples=2*\nsamples,smooth,domain=-\xmax:\xmax]
      plot(\t,{\A*int_one_ang(\k*\t,\lambd,\a)*int_two_ang(\k*\t,\lambd,\d)});
  \end{scope}
  
\end{tikzpicture}


% % DIFFRACTION - large slit - circular
% \def\H{5.4}     % total wall height
% \def\h{4.4}     % plane wave height
% \def\t{0.18}    % wall thickness
% \def\a{2.3}     % slit size
% \def\lambd{0.4} % wavelength
% \def\N{14}      % number of waves
% \def\dN{30}     % number shift
% \def\ang{6}    % angle of diffraction
% \begin{tikzpicture}
%   \message{Diffraction, circular^^J}
  
%   \def\lambd{0.4} % wavelength
  
%   % WAVES
%   \foreach \i [evaluate={\r=0.9*\i*\lambd;\R=(\i+\dN)*\lambd;}] in {1,...,\N}{
%     \ifodd\i
%       \draw[myblue,line width=0.8] (\r,{\R*sin(\ang)}) arc (\ang:-\ang:\R);
%     \else
%       \ifnumless{\i}{\N}{
%         \draw[myblue!80,line width=0.1] (\r,{\R*sin(\ang)}) arc (\ang:-\ang:\R);
%       }{}
%     \fi
%   }
%   \foreach \i [evaluate={\xp=-2*(\i-0.5)*\lambd; \xm=-2*(\i-1)*\lambd;}] in {1,...,2}{
%     \draw[myblue,line width=0.8] (\xp,-\h/2) -- (\xp,\h/2);
%     \draw[myblue,line width=0.1] (\xm,-\h/2) -- (\xm,\h/2);
%   }
  
%   % WALL
%   \fill[wall]
%     (\t/2,\a/2) rectangle (-\t/2,\H/2)
%     (\t/2,-\a/2) rectangle (-\t/2,-\H/2);
  
% \end{tikzpicture}


% % DIFFRACTION - large slit - straight
% \def\ang{10} % angle of diffraction
% \begin{tikzpicture}
%   \message{Diffraction, straight^^J}
  
%   % WAVES
%   \foreach \i [evaluate={\r=\i*\lambd;}] in {1,...,\N}{
%     \ifodd\i
%       \draw[myblue,line width=0.8] (\r,{\a/2+\r*sin(\ang)}) arc(\ang:0:\r) --++ (0,-\a) arc(0:-\ang:\r);
%     \else
%       \ifnumless{\i}{\N}{
%         \draw[myblue,line width=0.1] (\r,{\a/2+\r*sin(\ang)}) arc(\ang:0:\r) --++ (0,-\a) arc(0:-\ang:\r);
%       }{}
%     \fi
%   }
%   \foreach \i [evaluate={\xp=-2*(\i-0.5)*\lambd; \xm=-2*(\i-1)*\lambd;}] in {1,...,2}{
%     \draw[myblue,line width=0.8] (\xp,-\h/2) -- (\xp,\h/2);
%     \draw[myblue,line width=0.1] (\xm,-\h/2) -- (\xm,\h/2);
%   }
  
%   % WALL
%   \fill[wall]
%     (\t/2,\a/2) rectangle (-\t/2,\H/2)
%     (\t/2,-\a/2) rectangle (-\t/2,-\H/2);
  
% \end{tikzpicture}


% % DIFFRACTION - obstacle - straight
% \begin{tikzpicture}
%   \message{Diffraction, obstacle^^J}
  
%   % WAVES
%   \foreach \i [evaluate={\r=\i*\lambd;}] in {1,...,\N}{
%     \ifodd\i
%       \draw[myblue,line width=0.8]
%         (\r,\H/2) -- (\r,0.5*\a) arc (0:-\ang:\r)
%         (\r,-\H/2) -- (\r,-0.5*\a) arc (0:\ang:\r);
%     \else
%       \ifnumless{\i}{\N}{
%         \draw[myblue!80,line width=0.1]
%           (\r,\H/2) -- (\r,0.5*\a) arc (0:-\ang:\r)
%           (\r,-\H/2) -- (\r,-0.5*\a) arc (0:\ang:\r);
%       }{}
%     \fi
%   }
%   \foreach \i [evaluate={\xp=-2*(\i-0.5)*\lambd; \xm=-2*(\i-1)*\lambd;}] in {1,...,2}{
%     \draw[myblue,line width=0.8] (\xp,-\H/2) -- (\xp,\H/2);
%     \draw[myblue,line width=0.1] (\xm,-\H/2) -- (\xm,\H/2);
%   }
  
%   % WALL
%   \fill[wall] (-\t/2,-\a/2) rectangle (\t/2,\a/2);
  
% \end{tikzpicture}


% % DIFFRACTION - small slit
% \begin{tikzpicture}
%   \message{Diffraction, small slit^^J}
  
%   \def\H{5.5}     % total wall height
%   \def\h{4.0}     % plane wave height
%   \def\t{0.18}    % wall thickness
%   \def\a{0.68}    % slit distance
%   \def\N{7}       % number of waves
%   \def\lambd{0.4} % wavelength
%   \def\ang{40}
  
%   % WAVES
%   \foreach \i [evaluate={\Rp=2*\i*\lambd; \Rm=2*(\i+0.5)*\lambd;}] in {1,...,\N}{
%     \draw[myblue,line width=0.8] (-0.9*\lambd,0)++(\ang:\Rp) arc (\ang:-\ang:\Rp);
%     \ifnumless{\i}{\N}{
%       \draw[myblue!80,line width=0.1] (-0.9*\lambd,0)++(\ang:\Rm) arc (\ang:-\ang:\Rm);
%     }{}
%   }
%   \foreach \i [evaluate={\xp=-2*(\i-0.5)*\lambd; \xm=-2*(\i-1)*\lambd;}] in {1,...,2}{
%     \draw[myblue,line width=0.8] (\xp,-\h/2) -- (\xp,\h/2);
%     \draw[myblue,line width=0.1] (\xm,-\h/2) -- (\xm,\h/2);
%   }
  
%   % WALL
%   \fill[wall]
%     (\t/2,\a/2) rectangle (-\t/2,\H/2)
%     (\t/2,-\a/2) rectangle (-\t/2,-\H/2);
  
% \end{tikzpicture}


% DIFFRACTION - SINGLE SLIT - projection
\begin{tikzpicture}[
    nodal/.style={mylightgreen,dashed,very thin},
  ]
  \message{Diffraction, single slit^^J}
  
  \def\L{3.8}       % distance between walls
  \def\H{5.4}       % total wall height
  \def\h{2.8}       % plane wave height
  \def\t{0.15}      % wall thickness
  \def\a{0.8}       % slit distance
  \def\d{0.20}      % slit size
  \def\N{21}        % number of waves
  \def\dN{1}        % shift radius
  \def\A{1.6}       % amplitude
  \def\lambd{0.20}  % wavelength
  \def\R{\N*\lambd} % wave radius
  \def\Nfringes{2}  % number of nodal lines
  \def\nshades{70}  % number of shades
  \def\nsamples{40}
  \def\ang{40}
  
  % PLANE WAVES
  \foreach \i [evaluate={\xp=-2*(\i-0.5)*\lambd; \xm=-2*(\i-1)*\lambd;}] in {1,...,3}{
    \draw[myblue,line width=0.8] (\xp,-\h/2) -- (\xp,\h/2);
    \draw[myblue,line width=0.1] (\xm,-\h/2) -- (\xm,\h/2);
  }
  
  % WAVES
  \begin{scope}
    \clip (-\t/2,-\H/2) rectangle (\L,\H/2);
    \foreach \i [evaluate={\Rp=2*(\i+\dN)*\lambd; \Rm=2*(\i+\dN+0.5)*\lambd;}] in {1,...,\N}{
      \draw[myblue,line width=0.8] (-2.2*\lambd,0)++(\ang:\Rp) arc (\ang:-\ang:\Rp);
      \draw[myblue!80,line width=0.1] (-2.2*\lambd,0)++(\ang:\Rm) arc (\ang:-\ang:\Rm);
    }
    \foreach \m [evaluate={\ddy=0.1*\m;}] in {1,2,2.7}{ % mask at destructive interference
      \foreach \i [evaluate={\ym=\L/sqrt((\a/(\lambd*\m))^2-1); \dy=0.55*\i/\nshades;}]
        in {1,...,\nshades}{
        \fill[white,opacity=0.045] (0.05,0.005+\ddy) -- (\L,{\dy+\ym}) --++ (0,-2*\dy) -- (0.05,-0.005+\ddy);
        \fill[white,opacity=0.045] (0.05,0.005-\ddy) -- (\L,{\dy-\ym}) --++ (0,-2*\dy) -- (0.05,-0.005-\ddy);
      }
    }
  \end{scope}
  
  % WALL
  \fill[wall]
    (\t/2,\a/2) rectangle (-\t/2,\H/2)
    (\t/2,-\a/2) rectangle (-\t/2,-\H/2)
    (\L,-\H/2) rectangle (\L+\t,\H/2);
  
  % SHADES
  \begin{scope}[shift={(1.08*\L,0)}]
    \clip (0,-\H/2) rectangle (1.1*\A,\H/2);
    \fill[myshadow] (0,-\H/2) rectangle (\A,\H/2); % to fill seams
    \def\yz{\L/sqrt((\a/\lambd)^2-1)} %\L/sqrt((\a/\lambd)^2-1)
    \path [left color=myshadow,right color=myshadow,middle color=white,shading angle={180}]
      (0,{-\yz}) rectangle (\A,{\yz});
    \foreach \i [evaluate={
                  \n=\i;
                  \m=\i+1;
                  \yn=\L/sqrt((\a/(\lambd*\n))^2-1); %\L*\n*\lambd/\a
                  \ym=\L/sqrt((\a/(\lambd*\m))^2-1);
                  \dang=mod(\i,2)*180;
                 }] in {1,...,\Nfringes}{
      \path [left color=myshadow,right color=myshadow,middle color=myshadow!50,shading angle={\dang}]
        (0,\yn) rectangle (\A,\ym);
      \path [left color=myshadow,right color=myshadow,middle color=myshadow!50,shading angle={180+\dang}]
        (0,-\yn) rectangle (\A,-\ym);
    }
  \end{scope}
  
  % INTENSITY
  \begin{scope}[shift={(1.1*\L+1.1*\A,0)}]
    \draw[->,thick] (-0.08*\A,0) -- (1.3*\A,0) node[right=-2] {\(I\)};
    \draw[->,thick] (0,-0.52*\H) -- (0,0.54*\H) node[right] {\(x\)};
    \draw[myred,thick,variable=\y,samples=\nsamples,smooth,domain=-\H/2:\H/2]
      plot({\A*int_one(\y,\lambd,\a,\L)},\y);
    \foreach \i [evaluate={\y=\L/sqrt((\a/(\lambd*\i))^2-1);}] in {1,...,\Nfringes}{
      \tick{0,-\y}{180} node[right=-1,scale=0.9] {\(m=\i\)};
      \tick{0,\y}{180} node[right=-1,scale=0.9] {\(m=\i\)};
    }
  \end{scope}
  
\end{tikzpicture}


% % DIFFRACTION - ONE SLIT - PATH DIFFERENCE
% \begin{tikzpicture}
%   \message{Diffraction, path difference^^J}
  
%   \def\L{5.9}       % distance between walls
%   \def\H{4.3}       % total wall height
%   \def\f{0.95}      % fractional height of projection point
%   \def\ang{atan((\f*\H+\a/2)/\L/2)} % theta
%   \def\t{0.15}      % wall thickness
%   \def\a{2.5}       % slit distance
%   \coordinate (T) at (0,\a/2);
%   \coordinate (B) at (0,-\a/2);
%   \coordinate (L) at (0,0);
%   \coordinate (R) at (\L,0);
%   \coordinate (P) at (\L,\f*\H/2);
%   \coordinate (M) at ($(L)!(T)!(P)$);
  
%   % LINES
%   \draw[mygreen,thick] (T) -- (P); %node[midway,above=-1] {$r_1$};
%   \draw[mygreen,thick] (B) -- (P); %node[midway,below=3,right=6] {$r_2$}; %right=6,below right=-4
%   \draw[mygreen,thick] (L) -- (P);
%   \draw[dashed] (T) -- (B);
%   \draw[dashed,black!60] (L) -- (R);
%   \draw[mydarkred,dashed] (M) -- (T);
  
%   % ANGLES
%   \draw pic[mysmallarr,"$\theta'$",mydarkred,draw=mydarkred,angle radius=25,angle eccentricity=1.2]
%     {angle = B--T--M};
%   \draw pic[mysmallarr,"$\theta$",mydarkred,draw=mydarkred,angle radius=31,angle eccentricity=1.14]
%     {angle = R--L--P};
%   \rightAngle{T}{M}{P}{0.3}
  
%   % MEASURES
%   \draw[<->,black] (0,-0.47*\H) --++ (\L,0) node[midway,fill=white,inner sep=1] {$L$};
%   \draw[<->,black] (-2.1*\t,0) --++ (0,\a/2) node[midway,fill=white,inner sep=0.1] {$a/2$};
%   \draw[<->,black] (-2.1*\t,0) --++ (0,-\a/2) node[midway,fill=white,inner sep=0.1] {$a/2$};
%   \draw[<->,black] ([shift={({\ang-100}:0.1)}]L) -- ([shift={({\ang-100}:0.1)}]M)
%     node[midway,left=3,below right=-2]{$\frac{a}{2}\sin\theta'$};
%   \draw[<->,black] ([shift={(2.1*\t,0)}]P) -- ([shift={(2.1*\t,0)}]R)
%     node[midway,fill=white,inner sep=1]{$y$};
  
%   % WALL
%   \fill[wall]
%     (0,\a/2) rectangle (-\t,\H/2)
%     (0,-\a/2) rectangle (-\t,-\H/2)
%     (\L,-\H/2) rectangle (\L+\t,\H/2);
%   \fill[mygreen!80!black] (P) circle (0.3*\t) node[right=1,above left=-1] {P};
  
% \end{tikzpicture}


% % DIFFRACTION - ONE SLIT - PATH DIFFERENCE close-up
% \begin{tikzpicture}
%   \message{Diffraction, path difference (close-up)^^J}
  
%   \def\L{6.15}      % distance between walls
%   \def\l{4.0}       % path length
%   \def\H{4.3}       % total wall height
%   \def\f{0.9}       % fractional height of projection point
%   \def\t{0.15}      % wall thickness
%   \def\a{2.5}       % slit distance
%   \def\ang{26}      % angle
%   \coordinate (T) at (0,\a/2);
%   \coordinate (B) at (0,-\a/2);
%   \coordinate (L) at (0,0);
%   \coordinate (R) at (\L,0);
%   \coordinate (I) at ({\a/2/tan(\ang)},0);
  
%   % LINES
%   \draw[mygreen,thick] (T) --++ (\ang:.8*\l) coordinate (PT); %node[midway,below=1,above left=-2] {$r_1$};
%   \draw[mygreen,thick] (B) --++ (\ang:\l) coordinate (PB); %node[midway,left=2,below right=-1] {$r_2$};
%   \draw[mygreen,thick] (L) --++ (\ang:.9*\l) coordinate (PR);
%   \draw[dashed] (T) -- (B);
%   \draw[mydarkred,dashed] (T) -- ($(L)!(T)!(PR)$) coordinate (M);
%   %\draw[black!60,dashed] (T) --++ (.20*\L,0) coordinate (TR);
%   \draw[black!60,dashed] (L) --++ (.6*\L,0) coordinate (LR);
%   %\draw[black!60,dashed] (B) --++ (.20*\L,0) coordinate (BR);
  
%   % LINE END
%   \lineend{PT}{\ang+70}
%   \lineend{PR}{\ang+70}
%   \lineend{PB}{\ang+70}
  
%   % ANGLES
%   \draw pic[mysmallarr,"$\theta$",mydarkred,draw=mydarkred,angle radius=20,angle eccentricity=1.3]
%     {angle = B--T--M}; %\contour{white}{}
%   \draw pic[mysmallarr,"$\theta$",mydarkred,draw=mydarkred,angle radius=24,angle eccentricity=1.2]
%     {angle = LR--L--PR};
%   \draw pic[mysmallarr,"$\theta$",mydarkred,draw=mydarkred,angle radius=24,angle eccentricity=1.2]
%     {angle = LR--I--PB};
%   \rightAngle{T}{M}{PR}{0.3}
  
%   % MEASURES
%   \draw[<->,black] (-2.1*\t,0) --++ (0,\a/2) node[midway,fill=white,inner sep=0.1] {$a/2$};
%   \draw[<->,black] (-2.1*\t,0) --++ (0,-\a/2) node[midway,fill=white,inner sep=0.1] {$a/2$};
%   \draw[<->,black] ([shift={({\ang-90}:0.1)}]L) -- ([shift={({\ang-90}:0.1)}]M)
%     node[midway,left=5,below right=-1]{$\frac{a}{2}\!\sin\theta$};
  
%   % WALL
%   \fill[wall]
%     (0,\a/2) rectangle (-\t,\H/2)
%     (0,-\a/2) rectangle (-\t,-\H/2);
  
% \end{tikzpicture}

% DIFFRACTION - dispersion
\begin{tikzpicture}
  \message{Diffraction, dispersion^^J}
  \def\L{4.8}  % distance between walls
  \def\H{5.5}  % total wall height
  \def\l{5.5}  % second/third order ray length
  \def\t{0.18} % wall thickness
  \def\a{0.3}  % slit distance
  \def\N{200}  % number of rays in rainbow
  \def\A{1.2}  % band height
  \def\ym{0.128*\H} % minimum y coordinate of 2nd order (violet)
  \def\xb{1.08*\L}  % band x position
  
  % SHADE
  \fill[myshadow]
    (\xb,-\H/2) rectangle++ (\A,\H);
  \fill[left color=myshadow,right color=myshadow,middle color=white,shading angle=0]
    (\xb,-\ym) rectangle++ (\A,2*\ym);
  
  % RAYS, THIRD & FOURTH ORDER
  \message{  Third \& fourth order^^J}
  \begin{scope}
    \clip (0,-\H/2) rectangle++ (\L,\H);
    \foreach \i [evaluate={\f=\i/\N;\lamb=410.+\f*320.;\dy=0.051*\f;\ang=31+\f*16;}] in {0,...,\N}{
      \definecolor{tmpcol}{wave}{\lamb}
      \colorlet{mycol}[rgb]{tmpcol}
      \fill[mycol!3,line width=0.3] % third order
       (0.05, 0.028+\dy) -- ( 25+\ang:\l) -- ( 25+\ang+1:\l) -- (0.05, 0.030+\dy)
       (0.05,-0.028-\dy) -- (-25-\ang:\l) -- (-25-\ang+1:\l) -- (0.05,-0.030-\dy);
      \fill[mycol!14,line width=0.3]% second order
       (0.10, 0.028+\dy) -- ( \ang:\l) -- ( \ang+0.1:\l) -- (0.05, 0.030+\dy)
       (0.10,-0.028-\dy) -- (-\ang:\l) -- (-\ang-0.1:\l) -- (0.05,-0.030-\dy);
    }
  \end{scope}
  
  % RAYS, FIRST ORDER
  \message{  First order^^J}
  \draw[myblue,line width=4.6] (-0.35*\L,0) -- (0.05,0); % incoming white light
  \draw[white, line width=4.0] (-0.35*\L,0) -- (0.05,0); % incoming white light
  \draw[myblue,line width=1.8] (0.05,0) -- (\L,0);
  \draw[white, line width=1.5] (0.05,0) -- (\L,0);
  
  % RAYS, SECOND ORDER
  \message{  Second order^^J}
  \foreach \i [evaluate={\f=\i/\N;\lamb=410.+\f*320.;\dy=0.051*\f;
               \y=\ym+\f*0.25*\H;}]in {0,...,\N}{
    \definecolor{tmpcol}{wave}{\lamb}
    \colorlet{mycol}[rgb]{tmpcol}
    \fill[mycol!60,line width=0.3] % instead of smooth gradient, use many thin polygons
      (0.05, 0.028+\dy) -- (\L, \y) -- (\L, \y+0.01) -- (0.05, 0.030+\dy)
      (0.05,-0.028-\dy) -- (\L,-\y) -- (\L,-\y-0.01) -- (0.05,-0.030-\dy)
      (\xb, \y) rectangle++ (\A, 0.01)
      (\xb,-\y) rectangle++ (\A,-0.01);
  }
  
  % WALL
  \fill[wall]
    (\t/2,\a/2) rectangle (-\t/2,\H/2)
    (\t/2,-\a/2) rectangle (-\t/2,-\H/2)
    (\L,-\H/2) rectangle (\L+\t,\H/2);
  
\end{tikzpicture}

% DIFFRACTION - dispersion
\begin{tikzpicture}
    \message{Diffraction, dispersion^^J}
    \def\L{4.8}  % distance between walls
    \def\H{5.5}  % total wall height
    \def\l{5.5}  % second/third order ray length
    \def\t{0.18} % wall thickness
    \def\a{0.3}  % slit distance
    \def\N{200}  % number of rays in rainbow
    \def\A{1.2}  % band height
    \def\ym{0.128*\H} % minimum y coordinate of 2nd order (violet)
    \def\xb{1.08*\L}  % band x position
    
    % SHADE
    \fill[myshadow]
      (\xb,-\H/2) rectangle++ (\A,\H);
    \fill[left color=myshadow,right color=myshadow,middle color=white,shading angle=0]
      (\xb,-\ym) rectangle++ (\A,2*\ym);
    
    % RAYS, THIRD & FOURTH ORDER
    \message{  Third \& fourth order^^J}
    \begin{scope}
      \clip (0,-\H/2) rectangle++ (\L,\H);
      \foreach \i [evaluate={\f=\i/\N;\lamb=410.+\f*320.;\dy=0.051*\f;\ang=31+\f*16;}] in {0,...,\N}{
        \definecolor{tmpcol}{wave}{\lamb}
        \colorlet{mycol}[rgb]{tmpcol}
        \fill[mycol!3,line width=0.3] % third order
         (0.05, 0.028+\dy) -- ( 25+\ang:\l) -- ( 25+\ang+1:\l) -- (0.05, 0.030+\dy)
         (0.05,-0.028-\dy) -- (-25-\ang:\l) -- (-25-\ang+1:\l) -- (0.05,-0.030-\dy);
        \fill[mycol!14,line width=0.3]% second order
         (0.10, 0.028+\dy) -- ( \ang:\l) -- ( \ang+0.1:\l) -- (0.05, 0.030+\dy)
         (0.10,-0.028-\dy) -- (-\ang:\l) -- (-\ang-0.1:\l) -- (0.05,-0.030-\dy);
      }
    \end{scope}
    
    % RAYS, FIRST ORDER
    \message{  First order^^J}
    \draw[myblue,line width=4.6] (-0.35*\L,0) -- (0.05,0); % incoming white light
    \draw[white, line width=4.0] (-0.35*\L,0) -- (0.05,0); % incoming white light
    \draw[myblue,line width=1.8] (0.05,0) -- (\L,0);
    \draw[white, line width=1.5] (0.05,0) -- (\L,0);
    
    % RAYS, SECOND ORDER
    \message{  Second order^^J}
    \foreach \i [evaluate={\f=\i/\N;\lamb=410.+\f*320.;\dy=0.051*\f;
                 \y=\ym+\f*0.25*\H;}]in {0,...,\N}{
      \definecolor{tmpcol}{wave}{\lamb}
      \colorlet{mycol}[rgb]{tmpcol}
      \fill[mycol!60,line width=0.3] % instead of smooth gradient, use many thin polygons
        (0.05, 0.028+\dy) -- (\L, \y) -- (\L, \y+0.01) -- (0.05, 0.030+\dy)
        (0.05,-0.028-\dy) -- (\L,-\y) -- (\L,-\y-0.01) -- (0.05,-0.030-\dy)
        (\xb, \y) rectangle++ (\A, 0.01)
        (\xb,-\y) rectangle++ (\A,-0.01);
    }
    
    % WALL
    \fill[wall]
      (\t/2,\a/2) rectangle (-\t/2,\H/2)
      (\t/2,-\a/2) rectangle (-\t/2,-\H/2)
      (\L,-\H/2) rectangle (\L+\t,\H/2);
    
  \end{tikzpicture}


\end{document}