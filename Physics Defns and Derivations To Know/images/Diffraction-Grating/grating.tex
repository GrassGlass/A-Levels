% *Source: https://tex.stackexchange.com/a/599002
\documentclass[border=2mm]{standalone}
\usepackage    {tikz}
\usetikzlibrary{3d}    % For "canvas is..." options
\usetikzlibrary{babel} % There are conflicts between tikz and some babel packages
\usepackage{newpxtext, eulerpx}

% isometric axes
\pgfmathsetmacro\yx{1/sqrt(2)}
\pgfmathsetmacro\yy{1/sqrt(6)}
\pgfmathsetmacro\xy{sqrt(2/3)}

\newcommand{\rectangle}[6]% position (z), x,y,z dimensions, color, label 
{%
  \draw[canvas is xy plane at z=#1,fill=#5,fill opacity=0.8] (-#2,-#3)   rectangle (#2,#3);
  \draw[canvas is xz plane at y=#3,fill=#5]                  (-#2,#1-#4) rectangle (#2,#1);
  \draw[canvas is yz plane at x=#2,fill=#5]                  (-#3,#1-#4) rectangle (#3,#1);
  \node at (#2,-#3,#1) [above left,rotate=30] {#6};
}
\begin{document}
\begin{tikzpicture}[line cap=round,line join=round,%
                    x={(0 cm,\xy cm)},y={(-\yx cm,-\yy cm)},z={(\yx cm,-\yy cm)}]
% Dimensions
\def\pr{0.2} % Source, radius
\def\ph{2}   % Source, height
\def\rz{5}   % Grating, position (z)
\def\ra{1.5} % Grating, semi-dimension x
\def\rb{1.7} % Grating, semi-dimension y
\def\rc{0.1} % Grating, dimension z
\def\pz{10}  % Screen, postion (z)
\def\pa{\ra} % Screen, semi-dimension x
\def\pb{3.5}   % Screen, semi-dimension y
\def\pc{0.1} % Screen, dimension z
% Source
\draw[top color=gray] (-45:\pr)     --++ (0,0,\ph)  node[sloped,midway,above] {Source}
                  arc (-45:135:\pr) --++ (0,0,-\ph) arc (135:-45:\pr);
\draw[canvas is xy plane at z=\ph,fill=white] (0,0) circle (\pr);
% Single ray
\draw[blue!80!black,thick] (0,0,\ph) -- (0,0,\rz);
% Grating
\rectangle{\rz}{\ra}{\rb}{\rc}{brown!30}{Grating}
% Triple rays
\foreach\i in {-2.25,-1.25,-0.5,0,0.5,1.25,2.25}
{%
\pgfmathsetmacro\j{100*(1-(sqrt(abs(2*\i))/2.7))}
  \draw[blue!80!black!\j,thick] (0,0,\rz) -- (0,0.5*\pb*\i*0.8,\pz);
}
% Screen
\rectangle{\pz}{\pa}{\pb}{\pc}{white}{Screen}
\begin{scope}[canvas is xy plane at z=\pz]
  \foreach\i in {-2.25,-1.25,-0.5,0,0.5,1.25,2.25}
  {%
    \pgfmathsetmacro\j{100*(1-(sqrt(abs(2*\i))/2.7))}
    % \draw[dashed] (0,0.5*\pb*\i) --++   (-1.5*\pa,0);
    \fill[blue!80!black!\j]    (0,0.5*\pb*\i*0.8) circle (2pt);
    % \ifnum \i < 1
    %    \draw[<->] (-1.5*\pa,0.5*\pb*\i) --++ (0,0.5*\pb) node [midway,below] {$z$};
    % \fi
  }
\end{scope}
\draw[dashed] (-\ra,\rb,\rz-0.5*\rc)   -- (-\ra,\pb+1,\rz-0.5*\rc);
\draw[dashed] (-\pa,\pb,\pz-0.5*\rc)   -- (-\pa,\pb+1,\pz-0.5*\pc);
\draw[<->]    (-\ra,\pb+1,\rz-0.5*\rc) -- (-\pa,\pb+1,\pz-0.5*\pc) node[midway,below left] {\(D\)};
\node at (0,0,\pz) [above] {\(O\)};
\end{tikzpicture}
\end{document}