\setcounter{chapter}{-1}
\chapter{Foreword and Copyright}
\section*{Foreword}
\begin{itemize}
    \item These notes are meant to be summary notes for the core content of each subject, and for the key learning points I have consolidated from doing practice papers. They are \emph{not} meant to be a place for you to learn something from scratch.
    \item These notes are written, by me for me. So, don't expect uniform quality throughout. Notably, sections I wrote the earliest (time-wise) and/or have not needed to review much are less well-written and with less detail. Conversely, the sections I have revisited frequently have the best writing and highest detail.
    \item Expect some degree of spaghetti code. My preamble is in a mess, which I have never bothered to clean up Some parts of my code are not written in the correct way but held with duct tape --- I didn't always have the time nor patience to figure out the best way of doing things, given that JC can get busy. Nowadays, for newer latex projects, I use a far better preamble which also entails minimal spaghetti code.
    \item The above illustrates my point: my notes, for me. Everyone has different needs their notes should fulfil. That is why I have decided to make the underlying latex code open-source, instead of simply releasing the PDF. Furthermore, I hope that someone from each batch of students will update the code for their syllabus, for their own benefit and so that this repository will remain up-to-date for future generations of students.
    \item Also, there are bound to be many typos I have not found. So, don't use my notes blindly: my notes do not form a bible of absolute truth and accuracy.
\end{itemize}
\section*{Copyright}
\begin{itemize}
    \item A-Level H2 Physics 9749 Notes Copyright \textcopyright  2024 Shao Hong Yu. These notes are from my Github repository: \url{https://github.com/GrassGlass/A-Levels}.
    %* For those interested in editing this source latex code:
    %     This program is free software: you can redistribute it and/or modify
    %     it under the terms of the GNU General Public License as published by
    %     the Free Software Foundation, either version 3 of the License, or
    %     (at your option) any later version.

    %     This program is distributed in the hope that it will be useful,
    %     but WITHOUT ANY WARRANTY; without even the implied warranty of
    %     MERCHANTABILITY or FITNESS FOR A PARTICULAR PURPOSE.  See the
    %     GNU General Public License for more details.

    %     You should have received a copy of the GNU General Public License
    %     along with this program.  If not, see <https://www.gnu.org/licenses/>.

    % Also add information on how to contact you by electronic and paper mail.

    % If the program does terminal interaction, make it output a short
    % notice like this when it starts in an interactive mode:

    %     Grass' A-Levels Notes  Copyright (C) 2023  Grass
    %     This program comes with ABSOLUTELY NO WARRANTY; for details type `show w'.
    %     This is free software, and you are welcome to redistribute it
    %     under certain conditions; type `show c' for details.

    % The hypothetical commands `show w' and `show c' should show the appropriate
    % parts of the General Public License.  Of course, your program's commands
    % might be different; for a GUI interface, you would use an "about box".

    % You should also get your employer (if you work as a programmer) or school,
    % if any, to sign a "copyright disclaimer" for the program, if necessary.
    % For more information on this, and how to apply and follow the GNU GPL, see
    % <https://www.gnu.org/licenses/>.

    % The GNU General Public License does not permit incorporating your program
    % into proprietary programs.  If your program is a subroutine library, you
    % may consider it more useful to permit linking proprietary applications with
    % the library.  If this is what you want to do, use the GNU Lesser General
    % Public License instead of this License.  But first, please read
    % <https://www.gnu.org/licenses/why-not-lgpl.html>.
    \item The code (all .tex, .sty, etc files) written by me are released under \href{https://github.com/GrassGlass/A-Levels/blob/main/LICENSE}{The GNU General Public License v3.0}.
    \item All images (pdf, png, svg, jpg, etc files) written by me are release under \href{https://creativecommons.org/licenses/by/4.0/}{CC BY 4.0}.
    \item All files/images that are not written by me belong solely to their respective authors. These works are each licensed under \href{https://creativecommons.org/licenses/by/4.0/}{CC BY 4.0} or \href{https://creativecommons.org/licenses/by/3.0/}{CC BY 3.0}.
    \item If I forgot to credit any authors for their works, please let me know so that I can add that to my bibliography.
    \item If you are an author whose work I used in my repository and would like your accreditation to be modified, please let me know.
    \item Contact me: \href{mailto:shaohong00002@gmail.com}{shaohong00002@gmail.com} (Please do not email me unnecessarily.)
\end{itemize}
\section*{Requests to Users}
\begin{itemize}
    \item If you think that this repository will be helpful for you, please feel free to edit it to suit your needs.
    \item Following the terms of \href{https://github.com/GrassGlass/A-Levels/blob/main/LICENSE}{The GNU General Public License v3.0}, \href{https://creativecommons.org/licenses/by/4.0/}{CC BY 4.0}, and \href{https://creativecommons.org/licenses/by/3.0/}{CC BY 3.0}, please release any modified file open-source under the same license(s). Preferably, do it on GitHub.
    \item Please don't email me about typos: I don't want my email to be flooded.
\end{itemize}