\chapter{Work, Energy, and Power}
\begin{itemize}
    \item \emph{Work done} is defined as the product of a force and the displacement in the direction of the force.
    \item \emph{One joule of work} is defined as the work done by a one Newton force when its \emph{point of application} is displaced one metre in the direction of the force.
    \item \emph{Energy} is defined as the ability to do work.
    \item \emph{The Principle of Conservation of Energy} states that energy can neither be created or destroyed in \emph{any process}. It can be transformed from one form to another, and transferred from one body to another.
    \item Deriving \(E_k=\frac{1}{2}mv^2\): 
    \begin{enumerate}
        \item Consider a constant horizontal applied force \(F\) acting on an object of mass \(m\) travelling with initial velocity \(u\) to reach a final velocity \(v\) over a displacement \(s\). 
        \item For uniform acceleration, \(v^2=u^2+2as\) so \(as=\frac{1}{2}(v^2-u^2)\). Combined with Newton's Second Law, \(W=Fs=mas=\frac{1}{2}mv^2-\frac{1}{2}mu^2\). When the object starts from rest, \(u=0\). 
        \item By conservation of energy,\emph{ the work done by force \(F\) must be converted into the kinetic energy \(E_k\) of the object}. Hence, \(E_k=W=\frac{1}{2}mv^2-\frac{1}{2}m(0)^2=\frac{1}{2}mv^2\).
    \end{enumerate}
    \item \(E_k=\frac{p^2}{2m}\).
    \item The \emph{Work-Energy Theorem} states that the \emph{net} work done by \emph{external} forces acting on a particle is equal to the \emph{net} change in kinetic energy of the particle.
    \item Deriving \(E_p=mgh\):
    \begin{enumerate}
        \item Consider an object from the Earth's surface --- which is taken as the reference for zero gravitational potential energy --- raised up by a \emph{constant force \(F\) equal to and opposite to the weight \(mg\)} of the object such that the object moves up at \emph{constant velocity} to a height \(h\). 
        \item Thus, the object moves at constant speed so \(\Delta E_k=0\). Therefore, 
        \begin{align*}
            \Delta E_p&= W\\
            E_p-0&=Fs\\
            \qquad\qquad\qquad\qquad\qquad\qquad
            E_p&=mgh, \qquad\text{by Newton's Second Law.}
        \end{align*}
        where \(E_p\) is the gravitational potential energy at height \(h\) above the Earth's surface.
    \end{enumerate}
    \item Know how to derive \(\Delta E_p=\frac{1}{2}kx^2\) from area under graph.
    \item \emph{Power} is defined as the rate of doing work.
    \item Derive \(P=Fv\): \(P=\frac{\text{d}W}{\text{d}t}=\frac{\text{d}(Fs)}{\text{d}t}=F\frac{ds}{dt}=Fs\) for a constant force \(F\).
\end{itemize}